

\section{Introduction}\label{sec:introduction}


\noindent{\em Probabilistic programs with nondeterminism.} 
Randomness plays a fundamental role in many areas across science, and in 
computer science in particular.
In applications such as stochastic network protocols~\cite{BaierBook,prism},
randomized algorithms~\cite{RandBook,RandBook2}, 
security~\cite{BGGHS16:diff-privacy-coupling,BGHP16:diff-privacy-siglog} 
machine 
learning~\cite{LearningSurvey,G15},
the probabilistic behavior must be considered to faithfully model the underlying dynamic system.
The extension of classical imperative programs with \emph{random value generators}, 
that produce random values according to some desired probability distribution, 
naturally gives rise to probabilistic programs.
Along with probability, nondeterminism also plays a crucial role.
In particular in program analysis, for effective analysis of large programs,
all variables cannot be considered, and abstraction ignores some variables,
and the worst-case analysis is represented by adversarial nondeterminism.
Hence, probabilistic programs with nondeterminism have become an active and 
important research focus in program analysis. 
 


\smallskip\noindent{\em Termination problem.} 
In static analysis of programs the most basic, as well most important, 
liveness property is the {\em termination} problem.
While for non-probabilistic programs the termination question asks whether
an input program {\em always} terminates, for probabilistic programs 
the termination questions must account for the probabilistic behaviors. 
The most basic and fundamental extensions of the termination problem 
for probabilistic programs are:

\begin{compactenum}
\item \emph{Almost-sure termination.} 
The \emph{almost-sure termination} problem asks whether the program terminates with probability~1.


\item \emph{Positive termination.} 
The \emph{positive termination} problem asks whether the expected termination time is finite.
A related quantitative generalization of the positive termination question is to obtain 
asymptotic bounds on the expected termination time.

\end{compactenum}
While the positive termination implies almost-sure termination, the converse is not true 
(e.g., see Example~\ref{ex:infinite-time}).





\smallskip\noindent{\em Ranking functions and ranking supermartingales (RSMs).}
The key technique that applies for liveness analysis of non-probabilistic programs is 
the notion of {\em ranking functions}, which provides a sound and complete 
method for termination of non-probabilistic programs~\cite{rwfloyd1967programs}.
There exist a wide variety of approaches for construction 
of ranking functions for non-probabilistic programs~\cite{DBLP:conf/cav/BradleyMS05,DBLP:conf/tacas/ColonS01,DBLP:conf/vmcai/PodelskiR04,DBLP:conf/pods/SohnG91}.
The generalization of ranking functions to probabilistic programs is achieved through the
{\em ranking supermartingales (RSMs)}~\cite{SriramCAV,HolgerPOPL,CF17}.
The ranking supermartingales provide a powerful and automated approach for termination 
analysis of probabilistic programs, and algorithmic approaches for special cases such as 
linear and polynomial RSMs have also been considered~\cite{SriramCAV,CFNH16:prob-termination,CFG16,CNZ17}.



\smallskip\noindent{\em Practical limitations of existing approaches.} 
While an impressive set of theoretical results related to RSMs has been
established~\cite{SriramCAV,HolgerPOPL,CF17,CFNH16:prob-termination,CFG16,CNZ17},
 for probabilistic programs with nondeterminism the current approaches  
are only applicable to academic examples of variants of random walks. 
The key reason can be understood as follows: even for non-probabilistic programs while 
ranking functions are sound and complete, they do not necessarily provide a practical 
approach. This is because to prove termination, a witness in the form of a 
ranking function has to be computed: to do this automatically, ranking 
functions 
of a restricted form (such as linear ranking functions) have to be considered, 
and 1-dimensional ranking functions of a restricted type can only prove 
termination of a limited class of programs. 
In contrast, as a practical and scalable approach for non-probabilistic programs the 
notion of lexicographic ranking functions has been widely 
studied~\cite{CSZ13,ADFG10:lexicographic,GMR15:rank-extremal,BCIKP16:T2}. 
Algorithmic approaches for linear lexicographic ranking functions allow the 
termination analysis 
to be applicable to real-world non-probabilistic programs (after abstraction).
However both the theoretical foundations as well as practical approaches related to 
such lexicographic ranking functions are completely missing for probabilistic programs,
which we address in this work.







\smallskip\noindent{\em Our contributions.} 
In this work our main contributions range from theoretical foundations of lexicographic
RSMs, to algorithmic approaches for them, to experimental results
showing their applicability to programs.
We describe our main contributions below:
\begin{enumerate}

\item {\em Theoretical foundations.} 
First, we introduce the notion of lexicographic RSMs, and show that
such supermartingales ensure almost-sure termination (Theorem~\ref{thm:lexrsm-main} in 
Section~\ref{sec:lexicographic}). 
Our first result is a purely mathematical result that introduces a new concept, and 
proves almost-sure termination, that is independent of any probabilistic program.
Based on the mathematical result we show that for probabilistic programs with 
nondeterminism
the existence of a lexicographic RSM with respect to an invariant ensures
almost-sure termination (Theorem~\ref{thm:lexrsm-programs} in 
Section~\ref{sec:lex-programs}).

\item {\em Compositionality.} 
Second we study the compositional properties of lexicographic RSMs.
A key limitation of the previous approaches related to compositional 
RSMs~\cite{HolgerPOPL} is that it imposes a technical \emph{uniform 
integrability} conditions, which is hard to reason about automatically. We show 
(in Section~\ref{sec:compositional}) how 
lexicographic RSMs 
present an easy-to-automatize compositional approach for almost-sure 
termination of probabilistic programs.

\item {\em Algorithm.} We then consider algorithms for synthesis of lexicographic RSMs, 
and for efficient algorithms we consider nondeterministic probabilistic 
programs that are {\em affine} (i.e., the arithmetic operations are linear).
We present a polynomial-time algorithm for synthesis of lexicographic RSMs 
for affine programs (Theorem~\ref{thm:algo}).


\item {\em Asymptotic bounds.} 
In general, the existence of lexicographic RSMs does not imply positive
termination. 
In other words, we present an example (Example~\ref{ex:infinite-time}) where a 
lexicographic RSM exists ensuring
almost-sure termination, yet the expected termination time is infinite.
We then present a natural restriction under which the lexicographic 
RSMs not only imply positive termination, but even asymptotic bounds on the expected termination 
time can be derived from them (Theorem~\ref{thm:runtime-bound} and 
Corollary~\ref{col:runtime-progs}).


\item {\em Experimental results.} 
We present experimental results of our approach on realistic programs to show the applicability 
of our approach. 
To demonstrate the effectiveness of our approach we consider the benchmarks of non-probabilistic
programs from~\cite{ADFG10:lexicographic} which are obtained as abstraction of 
real-world programs, 
where 
lexicographic ranking functions were applied for termination analysis. 
We extend these benchmarks with probabilistic statements and apply lexicographic RSMs to these
programs. 
Our experimental results show that our approach can handle these programs very efficiently.


\end{enumerate}





