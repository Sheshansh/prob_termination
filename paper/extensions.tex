\section{Bounds on Expected Termination Time}

As shown in Example~\ref{ex:infinite-time}, already LinLexRSM maps are capable 
of proving almost-sure termination of programs whose expected termination time 
is infinite. However, it is often desirable to obtain bounds on expected 
runtime of a program. In this section, we present a LexRSM-based proof rule for 
obtaining bounds on expected runtime, and we show how to automatize the usage 
of this proof rule to obtain bounds on expected runtime in a 
subclass of \PP{}s.

As in the case of a.s. termination we start with general mathematical statement 
about LexRSMs. We define a restricted class of LexRSMs with \emph{bounded 
expected conditional increase} property. Recall from 
definition~\ref{def:lexrsm} that 
LexRSM for a stopping time $\stime$ is characterized by the possibility to 
a.s. partition, for each $i\in \Nset_0$, the set $\{\omega\in \Omega\mid 
\stime(\omega)>i \}$ into $d$ sets $L^i_1,\dots,L^i_d$ such that, intuitively, 
on $L^i_j$ the conditional expectation of $\vec{X}_{i+1}[j]$ given $\genfilt_i$ 
is smaller than $\vec{X}_i[j]$, and for all $j'<j$, on $L^i_{j'}$ the 
conditional expectation of $\vec{X}_{i+1}[j]$ given $\genfilt_i$ 
is no larger than $\vec{X}_i[j]$. This leaves the opportunity of conditional 
expectation of $\vec{X}_{i+1}[j]$ being larger than $\vec{X}_i[j]$ on 
$L^i_{j''}$ with $j''$. The conditional expected increase property bounds the 
possibility of this increase.

\begin{definition}
Let $\{\vec{X}_{i}\}_{i=0}^{\infty}$ be a 
$d$-dimensional LexRSM for some stopping time $\stime$, defined w.r.t. some 
filtration $\{\genfilt_i\}_{i=0}^{\infty}$. We say that 
$\{\vec{X}_{i}\}_{i=0}^{\infty}$ has \emph{$\vec{c}$-bounded expected 
conditional 
increase (ECI),} for 
some non-negative vector $\vec{c}\in\Rset^d$, if for each $i\in \Nset_0 $ and 
each $1\leq 
j \leq d $ it holds 
$\E[\vecseq{X}{i+1}{j}\mid \genfilt_i]\leq \vecseq{X}{i}{j}+\vec{c}[j]$ on 
$L^i_{j''}$, 
for all $j''>j$ 
(here $L^i_1,\dots,L^i_d$ are as in Definition~\ref{def:lexrsm}).
\end{definition}

For LexRSMs with $\vec{c}$-bounded ECI we have the following result. For 
simplicity, 
we formulate the result for 1-LexRSMs, though it is easy to prove analogous 
result for general $\eps$-LexRSMs, $\eps>0$, at the cost of obtaining less 
readable formula.

\begin{theorem}
\label{thm:runtime-bound}
Let $\{\vec{X}_{i}\}_{i=0}^{\infty}$ be a 
$d$-dimensional LexRSM with $c$-bounded ECI for some stopping time $\stime$. 
Then $\E[\stime]\leq  
\sum_{j=1}^{d}\E[\vecseq{X}{0}{j}]\cdot(\vec{c}[j]+1)^{d-j}$.
\end{theorem}
\begin{proof}
Denote $\noofdecrank_j(\omega)$ the number of steps $i$ in which $\omega\in 
L_j^i$. Since by Theorem~\ref{thm:lexrsm-main} the existence of LexRSM entails 
$\probm(\stime<\infty)=1$, the value $\noofdecrank_j(\omega)$ is a.s. finite. 
We prove that for each $1\leq j \leq d$ it holds $\E[\noofdecrank_j]\leq 
\vec{c}[j]\cdot\left(\sum_{j'<j}\E[\noofdecrank_{j'}]\right) + 
\E[\vecseq{X}{0}{j}].$ Since $\stime(\omega)=\sum_{1\leq j \leq d} 
\noofdecrank_j(\omega)$, for each $\omega\in \Omega$ (and hence, due to 
linearity of expectation $\E[\stime]=\sum_{1\leq j \leq d} 
E[\noofdecrank_j]$), the statement of the 
Theorem follows by an easy induction. 

To prove the required inequality, let $\nodecrank{k}{j}(\omega)$ be the number 
of steps $i$ \emph{within the first $k$} steps such that $\omega\in L_j^i$. We 
prove, by induction on $k$, that for each $k$ it holds 
$\E[\nodecrank{k}{j}]\leq 
\vec{c}[j]\cdot\left(\sum_{j'<j}\E[\nodecrank{k}{j'}] \right)+ 
\E[\vecseq{X}{0}{j}] - \E[\vecseq{X}{k}{j}]$. Once this is proved, the 
desired inequality follows by taking $k$ to $\infty$, since 
$\lim_{k\rightarrow \infty}\E[\nodecrank{k}{j}] = \E[\noofdecrank_j]$ and 
$\lim_{k\rightarrow \infty}\E[\vecseq{X}{k}{j}] \geq 0$.

The base case $k=0$ is simple as both sides of the inequality are zero. Assume 
that the inequality holds for some $k\geq 0$. We have 
$\E[\nodecrank{k+1}{j}]=\E[\nodecrank{k}{j}]+\probm{(L_j^{k})}$, so from 
induction hypothesis we get 
\begin{equation}
\label{eq:time1}
\E[\nodecrank{k+1}{j}]\leq 
\vec{c}[j]\cdot\left(\sum_{j'<j}\E[\nodecrank{k}{j'}] \right)+ 
\E[\vecseq{X}{0}{j}] - \E[\vecseq{X}{k}{j}] + \probm{(L_j^{k})}.\end{equation} 
Now denote 
$L^k_{<j} = L^k_1 \cup \dots\cup L^k_{j-1}$ and $L^k_{>j}= 
L^k_{j+1}\cup\dots\cup L^k_{d}$. We have  
$\E[\vecseq{X}{k}{j}] = \E[\vecseq{X}{k}{j}\cdot 
\indicator{ L_{<j}^k}] + \E[\vecseq{X}{k}{j}\cdot 
\indicator{ L_j^k}] + \E[\vecseq{X}{k}{j}\cdot 
	\indicator{L_{>j}^k}] \geq 
\E[\vecseq{X}{k+1}{j}\cdot 
	\indicator{ L_{<j}^k}] -\vec{c}[j]\cdot \probm(L_{<j}^k) + 
	\E[\vecseq{X}{k+1}{j}\cdot 
	\indicator{ L_{j}^k}] + \probm(L_{j}^k) + \E[\vecseq{X}{k+1}{j}\cdot 
	\indicator{ L_{>j}^k}]= \E[\vecseq{X}{k+1}{j}] -\vec{c}[j]\cdot 
	\probm(L_{<j}^k)+ \probm(L_{j}^k)$. Plugging this 
	into~\ref{eq:time1} yields
\begin{align*}
\E[\nodecrank{k+1}{j}]&\leq 
\vec{c}[j]\cdot\left(\sum_{j'<j}\E[\nodecrank{k}{j'}] \right)+ 
\E[\vecseq{X}{0}{j}] - \E[\vecseq{X}{k+1}{j}] + \vec{c}[j]\cdot 
\probm(L_{<j}^k)\\
&=\vec{c}[j]\cdot\left(\sum_{j'<j}\E[\nodecrank{k+1}{j'}] \right)+ 
\E[\vecseq{X}{0}{j}] - \E[\vecseq{X}{k+1}{j}].
\end{align*}
%
%\end{align}
\end{proof}

To transfer this mathematical result to probabilistic programs, we want to 
impose a restriction on LexRSM maps that ensures that all components of a 
LexRSM map have, from each reachable configuration, an expected one-step 
increase of at most $c$. Here $c$ can be a constant, but it can also be a value 
that depends on the initial configurations: this is to handle cases where some 
variables are periodically reset to a value related to the initial variable 
values, such as variable $z$ in Figure~\ref{fig:uniint2}. To this end, let 
$\program$ be a \PP{} with a pCFG $\pCFG_\program$ and let 
$\vec{\lem}=(\lem_1,\dots,\lem_d)$ be a 
$d$-dimensional $1$-LexRSM map for $\program$. Consider a $d$-dimensional 
vector 
$\vec{\bar{c}}=(\bar{c}_1,\dots,\bar{c}_d)$ whose each component is an 
expression over variables of 
$\program$. We say 
that $\vec{\lem}$ has 
$\vec{\bar{c}}$-bounded ECI w.r.t. invariant map $\inv$ if the following holds 
for each initial configuration $(\locinit,\vecinit)$ with $\vecinit\in 
\vecinitset$: for 
each 
configuration $(\loc,\vec{x})$ with $\vec{x}\in \inv(\loc)$ and generalized 
transition $\tilde{\tau}$ 
of 
$\pCFG_\program$ outgoing from $(\loc,\vec{x})$ it holds that if $j$ is 
the smallest index such that 
$\tilde{\tau}$ is $1$-ranked by $\lem_j$ from $(\loc,\vec{x})$, then for all 
$j'>j$ the gen. transition $\tilde{\tau}$ is $f$-ranked by $\lem_{j'}$ from 
$(\loc,\vec{x})$, where $f=-c_{j'}(\vecinit)$. From 
Theorem~\ref{thm:runtime-bound} we have the following:

\begin{corollary}
\label{col:runtime-progs}
Let $\program$ be a probabilistic program. Assume that there exists a 
$d$-dimensional $\eps$-LexRSM map $\vec{\lem}=(\lem_1,\dots,\lem_d)$ for 
$\program$ supported 
by some 
invariant map $\inv$, scuh that $\vec{\lem}$ has $\vec{\bar{c}}$-bounded ECI 
(w.r.t. $\inv$) 
for some vector of expressions $\vec{\bar{c}}=(\bar{c}_1,\dots,\bar{c}_d)$. 
Then under each scheduler $\sigma$ and for each initial valuation of program 
variables $\vecinit\in\vecinitset$ it holds $\E^{\sigma}_{\vecinit}[\ttime]\leq 
\sum_{j=1}^{d}\lem_j(\locinit,\vecinit)\cdot(\bar{c}_j(\vecinit))^{d-j}$.
\end{corollary}

\textbf{[PETR: FINISH THIS SECTION]}

