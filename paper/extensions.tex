\section{Bounds on Expected Termination Time}

As shown in Example~\ref{ex:infinite-time}, already LinLexRSM maps are capable 
of proving almost-sure termination of programs whose expected termination time 
is infinite. However, it is often desirable to obtain bounds on expected 
runtime of a program. In this section, we present a LexRSM-based method for 
obtaining bounds on expected runtime using a restricted class of LexRSMs.

As in the case of a.s. termination we start with general mathematical statement 
about LexRSMs. We define a restricted class of strict LexRSMs with \emph{bounded 
expected conditional increase} property. Recall from 
Definition~\ref{def:lexrsm} that strict
LexRSM for a stopping time $\stime$ is characterized by the possibility to 
a.s. partition, for each $i\in \Nset_0$, the set $\{\omega\in \Omega\mid 
\stime(\omega)>i \}$ into $n$ sets $L^i_1,\dots,L^i_d$ such that, intuitively, 
on $L^i_j$ the conditional expectation of $\vec{X}_{i+1}[j]$ given $\genfilt_i$ 
is smaller than $\vec{X}_i[j]$, and for all $j'<j$, on $L^i_{j'}$ the 
conditional expectation of $\vec{X}_{i+1}[j]$ given $\genfilt_i$ 
is no larger than $\vec{X}_i[j]$. This leaves the opportunity of conditional 
expectation of $\vec{X}_{i+1}[j]$ being larger than $\vec{X}_i[j]$ on 
$L^i_{j''}$ with $j''>j$. The expected conditional increase (ECI) property 
bounds the 
possibility of this increase. 

\begin{definition}
	\label{def:lexrsm-eci}
Let $\{\vec{X}_{i}\}_{i=0}^{\infty}$ be an 
$n$-dimensional strict LexRSM for some stopping time $\stime$, defined w.r.t. some 
filtration $\{\genfilt_i\}_{i=0}^{\infty}$. We say that 
$\{\vec{X}_{i}\}_{i=0}^{\infty}$ has \emph{$\vec{c}$-bounded expected 
conditional 
increase (ECI),} for 
some non-negative vector $\vec{c}\in\Rset^d$, if there exists an instance $(\{\vec{X}_{i=0}^{\infty},\{L_1^i,\dots,L_{n+1}^i\}_{i=0}^{\infty})$ of the strict LexRSM (i.e. $L^i_{n+1}=\emptyset$ for all $i$) such that for each $i\in \Nset_0 $ and 
each $1\leq 
j \leq n $ it holds 
$\E[\vecseq{X}{i+1}{j}\mid \genfilt_i]\leq \vecseq{X}{i}{j}+\vec{c}[j]$ on 
$L^i_{j''}$, 
for all $j''>j$ 
(here $L^i_1,\dots,L^i_n$ are as in Definition~\ref{def:lexrsm}).
\end{definition}

For strict LexRSMs with $\vec{c}$-bounded ECI we have the following result. For 
simplicity, 
we formulate the result for 1-LexRSMs, though it is easy to prove analogous 
result for general $\eps$-LexRSMs, $\eps>0$, at the cost of obtaining less 
readable formula.

\begin{theorem}
\label{thm:runtime-bound}
\label{THM:RUNTIME-BOUND}
Let $\{\vec{X}_{i}\}_{i=0}^{\infty}$ be an 
$n$-dimensional strict LexRSM with $c$-bounded ECI for some stopping time $\stime$. 
Then $\E[\stime]\leq  
\sum_{j=1}^{n}\E[\vecseq{X}{0}{j}]\cdot(1+\sum_{j'= j+1}^{n}\prod_{j'}^{n} 
\vec{c}[j'])$.
\end{theorem}
\begin{proof}
Fix an instance $(\{\vec{X}_{i=0}^{\infty},\{L_1^i,\dots,L_{n+1}^i\}_{i=0}^{\infty})$ satisfying Definition~\ref{def:lexrsm-eci}. Denote $\noofdecrank_j(\omega)$ the number of steps $i$ in which $\omega\in 
L_j^i$. Since by Theorem~\ref{thm:lexrsm-main} the existence of strict LexRSM entails 
$\probm(\stime<\infty)=1$, the value $\noofdecrank_j(\omega)$ is a.s. finite for all $1\leq j \leq n$. 
We prove that for each $1\leq j \leq n$ it holds $\E[\noofdecrank_j]\leq 
\vec{c}[j]\cdot\left(\sum_{j'<j}\E[\noofdecrank_{j'}]\right) + 
\E[\vecseq{X}{0}{j}].$ Since $\stime(\omega)=\sum_{1\leq j \leq n} 
\noofdecrank_j(\omega)$, for each $\omega\in \Omega$ (and hence, due to 
linearity of expectation $\E[\stime]=\sum_{1\leq j \leq n} 
E[\noofdecrank_j]$), the statement of the 
Theorem follows by an easy induction. 

To prove the required inequality, let $\nodecrank{k}{j}(\omega)$ be the number 
of steps $i$ \emph{within the first $k$} steps such that $\omega\in L_j^i$. We 
prove, by induction on $k$, that for each $k$ it holds 
$\E[\nodecrank{k}{j}]\leq 
\vec{c}[j]\cdot\left(\sum_{j'<j}\E[\nodecrank{k}{j'}] \right)+ 
\E[\vecseq{X}{0}{j}] - \E[\vecseq{X}{k}{j}]$. Once this is proved, the 
desired inequality follows by taking $k$ to $\infty$, since 
$\lim_{k\rightarrow \infty}\E[\nodecrank{k}{j}] = \E[\noofdecrank_j]$ and 
$\lim_{k\rightarrow \infty}\E[\vecseq{X}{k}{j}] \geq 0$. The inductive proof is 
somewhat technical and deferred to \AppendixMaterial.
%
%The base case $k=0$ is simple as both sides of the inequality are zero. Assume 
%that the inequality holds for some $k\geq 0$. We have 
%$\E[\nodecrank{k+1}{j}]=\E[\nodecrank{k}{j}]+\probm{(L_j^{k})}$, so from 
%induction hypothesis we get 
%\begin{equation}
%\label{eq:time1}
%\E[\nodecrank{k+1}{j}]\leq 
%\vec{c}[j]\cdot\left(\sum_{j'<j}\E[\nodecrank{k}{j'}] \right)+ 
%\E[\vecseq{X}{0}{j}] - \E[\vecseq{X}{k}{j}] + \probm{(L_j^{k})}.\end{equation} 
%Now denote 
%$L^k_{<j} = L^k_1 \cup \dots\cup L^k_{j-1}$ and $L^k_{>j}= 
%L^k_{j+1}\cup\dots\cup L^k_{d}$. We have  
%$\E[\vecseq{X}{k}{j}] = \E[\vecseq{X}{k}{j}\cdot 
%\indicator{ L_{<j}^k}] + \E[\vecseq{X}{k}{j}\cdot 
%\indicator{ L_j^k}] + \E[\vecseq{X}{k}{j}\cdot 
%	\indicator{L_{>j}^k}] \geq 
%\E[\vecseq{X}{k+1}{j}\cdot 
%	\indicator{ L_{<j}^k}] -\vec{c}[j]\cdot \probm(L_{<j}^k) + 
%	\E[\vecseq{X}{k+1}{j}\cdot 
%	\indicator{ L_{j}^k}] + \probm(L_{j}^k) + \E[\vecseq{X}{k+1}{j}\cdot 
%	\indicator{ L_{>j}^k}]= \E[\vecseq{X}{k+1}{j}] -\vec{c}[j]\cdot 
%	\probm(L_{<j}^k)+ \probm(L_{j}^k)$. Plugging this 
%	into~\ref{eq:time1} yields
%\begin{align*}
%\E[\nodecrank{k+1}{j}]&\leq 
%\vec{c}[j]\cdot\left(\sum_{j'<j}\E[\nodecrank{k}{j'}] \right)+ 
%\E[\vecseq{X}{0}{j}] - \E[\vecseq{X}{k+1}{j}] + \vec{c}[j]\cdot 
%\probm(L_{<j}^k)\\
%&=\vec{c}[j]\cdot\left(\sum_{j'<j}\E[\nodecrank{k+1}{j'}] \right)+ 
%\E[\vecseq{X}{0}{j}] - \E[\vecseq{X}{k+1}{j}].
%\end{align*}
%
%\end{align}
\end{proof}

To transfer this mathematical result to probabilistic programs, we want to 
impose a restriction on LexRSM maps that ensures that all components of a 
LexRSM map have, from each reachable configuration, an expected one-step 
increase of at most $c$. Here $c$ can be a constant, but it can also be a value 
that depends on the initial configurations: this is to handle cases where some 
variables are periodically reset to a value related to the initial variable 
values, such as variable $y$ in Figure~\ref{fig:uniint4}. To this end, let 
$\program$ be a \PP{} with a pCFG $\pCFG_\program$ and let 
$\overrightarrow{\lem}=(\lem_1,\dots,\lem_n)$ be an 
$n$-dimensional $1$-LexRSM map for $\program$. Consider an $n$-dimensional 
vector 
$\vec{\bar{c}}=(\bar{c}_1,\dots,\bar{c}_n)$ whose each component is an 
expression over variables of 
$\program$. We say 
that $\overrightarrow{\lem}$ has 
$\vec{\bar{c}}$-bounded ECI w.r.t. invariant map $\inv$ if the following holds 
for each initial configuration $(\locinit,\vecinit)$ with $\vecinit\in 
\vecinitset$: for 
each 
configuration $(\loc,\vec{x})$ with $\vec{x}\in \inv(\loc)$ and generalized 
transition $\tilde{\tau}$ 
of 
$\pCFG_\program$ outgoing from $(\loc,\vec{x})$ it holds that if $j$ is 
the smallest index such that 
$\tilde{\tau}$ is $1$-ranked by $\lem_j$ from $(\loc,\vec{x})$, then for all 
$j'>j$ the gen. transition $\tilde{\tau}$ is $f$-ranked by $\lem_{j'}$ from 
$(\loc,\vec{x})$, where $f=-c_{j'}(\vecinit)$. From 
Theorem~\ref{thm:runtime-bound} we have the following:

\begin{corollary}
\label{col:runtime-progs}
Let $\program$ be a probabilistic program. Assume that there exists an 
$n$-dimensional $\eps$-LexRSM map $\overrightarrow{\lem}=(\lem_1,\dots,\lem_d)$ 
for 
$\program$ supported 
by some 
invariant map $\inv$, such that $\overrightarrow{\lem}$ has 
$\vec{\bar{c}}$-bounded ECI 
(w.r.t. $\inv$) 
for some vector of expressions $\vec{\bar{c}}=(\bar{c}_1,\dots,\bar{c}_n)$. 
Then under each scheduler $\sigma$ and for each initial valuation of program 
variables $\vecinit\in\vecinitset$ it holds $\E^{\sigma}_{\vecinit}[\ttime]\leq
\sum_{j=1}^{n}\lem_j(\locinit,\vecinit)\cdot(1+\sum_{j'= j+1}^{n}\prod_{j'}^{n} 
\bar{c}_{j'}(\vecinit)).$
%\sum_{j=1}^{n}\lem_j(\locinit,\vecinit)\cdot(\bar{c}_j(\vecinit))^{n-j}$.
\end{corollary}

\begin{remark}
Previous works testify that analogous ``bounded expected change'' constraints 
naturally emerge when one aims to obtain additional information about termination 
time of probabilistic programs~\cite{HolgerPOPL,CFNH16:prob-termination,CNZ17}.
However all previous works consider bounded expected change to use well-known 
tail-bound inequalities for martingales such as Azuma's inequality to obtain 
probability bounds on termination time, but not asymptotic bounds on expected
termination time.
In contrast, the major novelty of our bounded ECI condition is that it we show it can be used to 
obtain asymptotic bounds on expected termination time, and in particular, they 
can be used to obtain polynomial bounds on expected runtime using \emph{linear} 
supermartingales.
\end{remark}

\begin{example}
Consider the program in Figure~\ref{fig:uniint4}, with initial condition 
$x_{\mathit{init}}\geq 0$. For the program we have an 
invariant map $\inv$ with $\inv(\loc_0)=x\geq -1$, $\inv(\loc_1)=\inv(\loc_4) = 
x\geq 0 
\wedge z\geq 0$, $\inv(\loc_2)=\inv(\loc_3)= x\geq 0 \wedge y \geq 0
\wedge z\geq 0$, and $\inv(\loc^{\lout})=\mathit{true}$. Further, there exists 
the following 
$3$-dimensional 
$1$-LinLexRSM map $\overrightarrow{\lem}$ supported by $\inv$: 
$\overrightarrow{\lem}(\loc_0)=(2x+3,0,2)$, 
$\overrightarrow{\lem}(\loc_1)=(2x+3,0,1)$, 
$\overrightarrow{\lem}(\loc_2)=(2x+2,y,1)$, 
$\overrightarrow{\lem}(\loc_3)=(2x+2,y,0)$, 
$\overrightarrow{\lem}(\loc_4)=(2x+2,0,0)$, and 
$\overrightarrow{\lem}(\loc^{\lout})=(0,0,0)$. It is 
easy to check that $\overrightarrow{\lem}$ is has $(0,z,2)$-bounded ECI. 
Applying 
Corollary~\ref{col:runtime-progs} yields $\E^{\sigma}_{\vecinit}[\ttime]\leq 
(2x_{\mathit{init}} + 
3)\cdot(1+3z_{\mathit{init}})+0+2= 6x_{\mathit{init}}z_{\mathit{init}} + 
2x_{\mathit{init}} + 9z_{\mathit{init}}+5$. Note that this quadratic bound is 
asymptotically optimal for the program.
\end{example}

\begin{remark}[Polynomial bounds]\label{rem:poly}
Now consider that we have synthesized a $1$-LexRSM map $\overrightarrow{\lem}$ 
and we want 
to check if there exists $\bar{\vec{c}}$ such that $\overrightarrow{\lem}$ has 
$\bar{\vec{c}}$-bounded ECI. In the linear setting (i.e. the program is an 
\APP{}, masp $\lem$ and $\inv$ are linear, and we seek $\vec{\bar{c}}$ which is 
a vector of affine expressions) we can encode the existence of $\vec{\bar{c}}$ 
into a system of linear inequalities in a similar way as the existence of a 
linear LexRSM maps was encoded in Section~\ref{sec:lex-programs}. I.e., we 
set up a linear template with unknown coefficients for each component of 
$\vec{\bar{c}}$ and using Farkas's lemma we set up a system of linear 
constraints, which includes the unknown coefficients as variables, encoding the 
fact that $\overrightarrow{\lem}$ has $\bar{\vec{c}}$-bounded ECI. In this way, 
we can check in polynomial time if 
Corollary~\ref{col:runtime-progs} can be applied to $\overrightarrow{\lem}$, 
and if yes, 
we can synthesize the witness vector $\bar{\vec{c}}$. Since $\bar{\vec{c}}$ 
consists of affine expressions, Corollary~\ref{col:runtime-progs} provides a 
polynomial (in the size of 
initial variable valuation) upper bound on expected runtime.
\end{remark}

%\textbf{[PETR: FINISH THIS SECTION]}

