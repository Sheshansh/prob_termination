\section{Compositionality of Ranking Supermartingales Revisited}
\label{sec:compositional}
In this section we discuss compositional aspects. We start with the 
one-dimensional case, and then present the multidimensional case.

\subsection{One-Dimensional Compositional Proofs of Almost-Sure Termination}

Compositionality in the context of termination proving means providing the 
proof of termination step-by-step, handling one loop at a time, rather than 
attempting to construct the proof (in our case, a LexRSM) at 
once~\cite{KSTW10:compositional-transition-invariants}. 
%In non-probabilistic setting, the 
%relationship between lexicographic ranking functions and compositional 
%termination proving is well established, as well as the importance of 
%compositional proofs for efficient implementation. 
In the context of 
probabilistic programs, the work~\cite{HolgerPOPL} attempted to provide a 
compositional notion of almost-sure termination proof based on the 
\emph{probabilistic variant rule (V-rule),} which we explain in a more detail 
below. However, for the method to work,~\cite{HolgerPOPL} imposes a 
technical~\emph{uniform integrability} condition, whose checking is hard to 
automatize. In this 
section we show that using our insights into LexRSMs we 
can obtain a different notion of a probabilistic V-rule which is sound 
without any additional assumptions, and which can be used to compositionally 
prove termination 
of programs that the previous method cannot handle. 

%The reason for this attempt was to provide a method of 
%proving a.s. termination for programs where 1-dimensional ranking 
%supermartingales are not sufficient. Since our LexRSMs were conceived with a 
%similar motivation, we devote this section to a thorough comparison of our 
%approach with the one of~\cite{HolgerPOPL}. 

%There are two main features of LexRSMs compared with the previous approach, 
%which we call a PVSM (probabilistic V-rule supermartingale) approach. First, 
%the PVSM approach is only applicable to programs in which termination of each 
%loop can be witnessed by a suitable 1-dimensional ranking supermartingale. But 
%there are already single-loop programs for which no 1-dimensional ranking 
%supermartingale exists but a LexRSM does, see the following example.
%
%\textbf{[PETR: ADD EXAMPLE]}
%
%%The need for multi-dimensional ranking arises within loops that might go 
%%through several ``computation phases'' before terminating, and loops 
%%containing 
%%non-deterministic choices.
%
%Second, as already noted in~\cite{HolgerPOPL}, the probabilistic V-rule is not 
%sound in general. In order to rectify this issue, in~\cite{HolgerPOPL} they 
%impose an additional constraint of \emph{uniform integrability} on ranking 
%supermartingales that prove termination of individual loops, 
%under which the probabilistic V-rule is shown to be sound. Apart from uniform 
%integrability being somewhat restrictive in itself, in~\cite{HolgerPOPL} it is 
%argued that proving uniform integrability is beyond the capability of 
%state-of-the-art automated theorem provers. As a substitute for 
%these,~\cite{HolgerPOPL} introduces a type system that can be used to 
%automatically prove 
%uniform integrability of ranking supermartingales for a restricted class of 
%programs. In particular, the method cannot handle programs in which 
%termination 
%is controlled by a variable that can be modified by a non-constant value in 
%some sub-program, e.g. 
%variable $x$ which appears in an assignment of the form $x:=x-y$. 

Let $\program$ be a \PP{} of the form $\textbf{while } \Psi \textbf{ do } 
\programbody \textbf{ od}$, and let $\pCFG_{\program}$ be the associated pCFG, whose set of locations we denote by $\locs$. We denote by $\loops(\program)$ the set of all locations of $\pCFG_{\program}$ that belong to a sub-pCFG of $\pCFG_{\program}$ corresponding to some nested loop of $\program$. We also define $\slice(\program)$ to be the set $\locs\setminus\loops(\program)$ of locations that do not belong to any nested sub-loop.  A formal definition of 
both functions is given in \AppendixMaterial, we illustrate them in the 
following example.
%We denote by $\loops(\program)$ the set of all 
%loops in $\program$ there are on the top level of $\programbody$, i.e. they are 
%nested loops of $\program$ nested directly "below" the outer loop. We define an 
%un-nested slice of $\program$ to be a \PP{} $\slice(\program)$ obtained from 
%$\program$ by removing all loops in $\loops(\program)$ and replacing them with 
%skip statements. We also define the set 
%$\alloops(\program)$ of all loops in $\program$, i.e. 
%$\alloops(\program):=\loops(\program)\cup \{\loops(\program')\mid \program'\in 
%\loops(\program)\}$. The above definitions do not specify the initialization 
%preamble for the sub-programs $\program'\in\alloops(\program)$: we can deem 
%this pre-amble empty, i.e. consider these sub-programs with any initial 
%valuation, or, provided that some invariant map $\inv$ for $\program$ is 
%given, we can set the set of possible initial valuations of a sub-program 
%$\program'$ to be the set $\inv(\loc^{\lin}_{\program'})$, where 
%$\inv(\loc^{\lin}_{\program'})$ is the location of $\program$ corresponding to 
%the head of the loop $\program'$.
%A formal definition of 
%both functions is given in \AppendixMaterial, we illustrate them in the 
%following example.

%\textbf{[PETR: INSERT EXAMPLE]}
\begin{example}
Consider the program $\program$ in Figure~\ref{fig:uniint4} and its associated 
pCFG. Then $\loops(\program)=\{\loc_2,\loc_3\}$ and 
$\slice(\program)=\{\loc_0,\loc_1,\loc_4,\loc^{\lout}\}$.
\end{example}

Given an invariant map $\inv$, we say that a 1-dimensional measurable map 
$\lem$ for $\pCFG_{\program}$ is $\eps$-$\inv$-ranking/unaffecting in location $\loc$, if 
for each $\vec{x}\in\inv(\loc)$  each generalized transition~$\tilde{\tau}$ 
outgoing from $\loc$ is $\eps$-ranked/unaffected by $\lem$. 

We recall the notion of compositional ranking supermartingale as introduced 
in~\cite{HolgerPOPL}. We call it a PV supermartingale, as it is based on so 
called probabilistic variant rule. Due to differences in syntax and semantics, 
the definition is syntactically slightly different from~\cite{HolgerPOPL}, but 
the essence is the same. A measurable map $\lem$ is propositionally linear, if 
each function $\lem(\loc)$ is of the form $\indicator{G_1}\cdot E_1 + \cdots 
\indicator{G_k}\cdot E_k$, where each $\indicator{G_i}$ is an indicator 
function of some polyhedron and each $E_i$ is a linear expression.

\begin{definition}[PV-supermartingale {\cite[Definition 7.1.]{HolgerPOPL}}]
A 1-dimensional propositionally linear map $\lem$ is a PV supermartingale (PVSM) for a program $\program$ supported by an invariant map $\inv$ if there exists $\eps>0$ such that $\lem$ is $\eps$-$\inv$-ranking an $\inv$-non-negative in each location $\loc\in\slice{(\program)}$ and $\inv$-unaffected in each $\loc\in\loops(\program)$.
\end{definition}

As a matter of fact, the condition that $\lem$ should be non-negative in locations of $\slice{(\program)}$ is not explicitly mentioned in \cite{HolgerPOPL}. However, it is implicitly used in some of the proofs and one can easily construct an example where, if the non-negativity in $\slice{(\program)}$ is not required, the Theorem~\ref{thm:holger-comp} below, which also comes from~\cite{HolgerPOPL}, does not hold. Hence, we state the condition explicitly.

In~\cite{HolgerPOPL} they show that even if all nested loops were already 
proved to terminate a.s. and there is a PVSM for the program, then the program 
itself might not terminate a.s. Then they impose a \emph{uniform integrability} 
constraint on the PVSM under which a 
PVSM together with a proof of a.s. termination of each nested sub-loop of 
$\program$ entails termination of the whole program $\program$. 
%they 
%impose an additional constraint of \emph{uniform integrability} on ranking 
%supermartingales that prove termination of individual loops, 
%under which the probabilistic V-rule is shown to be sound. 
Uniform integrability is a deep 
concept from probability and measure theory: a sequence $X_0,X_1,X_2,\dots$ of 
random variables is uniformly integrable if for each $\delta>0$ there exists an 
$K\in \Nset$ such that for all $n\geq 0$ it holds $\E[|X_n|\cdot\indicator{X_n 
	\geq K}]\leq \delta$.
Apart from uniform 
integrability being somewhat restrictive in itself, in~\cite{HolgerPOPL} it is 
argued that proving uniform integrability is beyond the capability of 
state-of-the-art automated theorem provers. As a substitute for 
these,~\cite{HolgerPOPL} introduces a type system that can be used to 
automatically prove 
uniform integrability of ranking supermartingales for a restricted class of 
programs.
We do not repeat the precise definition of the typesystem here, we just say 
that a PVSM satisfying the condition imposed by the type system 
\emph{typechecks} correctly. 
In~\cite{HolgerPOPL} the following was proved:

\begin{theorem}[\cite{HolgerPOPL}]
\label{thm:holger-comp}
Let $\program$ be a \PP{} of the form $\textbf{while } \Psi \textbf{ do } 
\programbody \textbf{ od}$. Assume that each nested loop of $\program$ 
terminates almost surely from each reachable configuration, and that there 
exists a PVSM for $\program$ that typechecks correctly. Then $\program$ 
terminates almost surely.
\end{theorem}

%, we just explain its motivation and give examples where it fails, i.e. 
%examples where $\program$ terminates a.s. (and hence the same holds also for 
%its nested loops) and a PV martingale exists, but no PVSM typechecks.  

%The purpose of the typesystem is, in principle, to ensure that the stochastic 
%process which, in each time step, returns the current value of the PVSM, is 
%uniformly integrable inside the nested loops.  As noted in~\cite{HolgerPOPL}, 
%proofs of uniform 
%integrability are hard to (semi)-automatize, and hence they devise the 
%typechecking as a sound (but not complete) method of proving uniform 
%integrability. This is illustrated in the following example.
The intricacies of uniform integrability are shown in the following example.

\begin{example}
\label{ex:uniform}
\label{EX:UNIFORM}
Consider the two \APP{}s in Figure~\ref{fig:uniint}, that differ only in one 
coefficient in the 
assignment on line 5. For the inner loop there exists (in both cases) a 
1-dimensional linear ranking supermartingale whose value in each location is 
equal to $c+d_{\loc}$, where $d_{\loc}$ is a location-specific constant. Since 
the expected change of $c$ in each loop step is $-0.5$, this is indeed a LRSM. 
Also, in both cases, a LEM of the form $x+d_{\loc}'$, again for some suitable 
location-specific constants $d_{\loc}'$, is a PVSM for the outer loop, as $x$ 
decreases and is non-negative within the outer loop and its expected change is 
non-negative within the inner loop (more precisely, the inner-loop expected 
change of $x$ zero in the left program and $-\frac{x}{4} - \frac{3}{4}$ in the 
right 
program). 
However, the variable $x$ is uniformly integrable within the inner loop of the 
right program while for the left program this does not hold: we show this 
in~\AppendixMaterial. The example shows that proving uniform integrability 
requires intricate reasoning about quantitative behaviour of the program. 
Moreover, as shown below, none of the two programs have a PVSM that typechecks.

% compute the distribution of 
%$x$ in this stand-alone program after $n$ steps. In both cases, $x$ has value 
%$0$ with probability 
%$1-\frac{1}{2^n}$, as this is the probability of setting $x$ to zero in the 
%first $n$ steps, after which the loop terminates. With the remaining 
%probability $\frac{1}{2^n}$ the value of $x$ is $2^n \cdot x_0 - 2^{n+1}+2$ in 
%the left program and $(\frac{3}{2})^n \cdot x_{0} -3\cdot((\frac{3}{2})^n +1)$ 
%in the right program, where $x_0$ is the value of $x$ upon entering the inner 
%loop (both equations are obtained by solving a simple linear recurrence). 
%Hence, for the right program, the expectation of $x$ after $n$ steps of the 
%inner loop (viewed as a stand-alone program) tends to zero as $n\rightarrow 
%\infty$, which in particular shows uniform integrability. However, in left 
%inner loop, $\E[|X_n|\cdot\indicator{X_n 
%	\geq }]\geq 1$ 
\end{example}
%However, as we show below, for none of these programs there exists a PVSM that 
%typechecks.

\lstset{language=affprob}
\lstset{tabsize=2}
\newsavebox{\uniinto}
\begin{lrbox}{\uniinto}
\begin{lstlisting}[mathescape]
while $x\geq 0$ do
	$c:=1$;
	while $x\geq 1$ and $c\geq 1$ do
		if prob(0.5) then $x:=0$
			else $x:=2(x-1)$ fi;
		if prob(0.5) then $c:=0$ 
			else skip fi
	od;
	$x:=x-1$
od
\end{lstlisting}
\end{lrbox}
\newsavebox{\uniintt}
\begin{lrbox}{\uniintt}
	\begin{lstlisting}[mathescape]
while $x\geq 0$ do
	$c:=1$;
	while $x\geq 1$ and $c\geq 1$ do
		if prob(0.5) then $x:=0$
			else $x:=\frac{3}{2}(x-1)$ fi;
		if prob(0.5) then $c:=0$ 
			else skip fi
	od;
	$x:=x-1$
od
	\end{lstlisting}
\end{lrbox}
\begin{figure}[t]
	%%\centering
	\usebox{\uniinto}
	\hspace{1cm}
	\usebox{\uniintt}
\caption{Examples of programs with (right) and without (left) uniformly 
integrable PVSMs.}
\label{fig:uniint}
\end{figure}

\lstset{language=affprob}
\lstset{tabsize=2}
\newsavebox{\unintthree}
\begin{lrbox}{\unintthree}
	\begin{lstlisting}[mathescape]
while $x\geq 0$ do
	while $y\geq 0$ do
		$z:=x$;
		while $z\geq 0$ do
			$z:=z-1$;
			$x:=x-1$
		od;
		$y:=y-1$
	od;
	$x:=x+$sample$(\mathit{Uniform[-3,1]})$
od
	\end{lstlisting}
\end{lrbox}
\newsavebox{\unintfour}
\begin{lrbox}{\unintfour}
	\begin{lstlisting}[mathescape]
while $x\geq 0$ and $z \geq 0 $ do
	$y:=z$;
	while $y\geq 3$ do
		$y:=y+$sample$(\mathit{Uniform[-3,1]})$
	od;
	$x:=x-1$
od
	\end{lstlisting}
\end{lrbox}
\begin{figure}[t]
	%%\centering
	\begin{subfigure}{0.45\textwidth}
		\usebox{\unintfour}
		\vspace{0.2cm}
		\begin{tikzpicture}[x=1.8cm,y=1.4cm]
		\node[det] (init) at (0,0) {$\loc_0$};
		\node[det] (fin) at (0,-1) {$\loc^{\lout}$};
		\node[det] (as1) at (1.2,0) {$\loc_1$};
		\node[det] (inner) at (2,0) {$\loc_2$};
		\node[det] (as2) at (2,-1) {$\loc_3$};
		\node[det] (as3) at (3,0) {$\loc_4$};
%		
		\draw[tran] (init) to node[font=\scriptsize,draw, fill=white, 
		rectangle,pos=0.6] {$x<0 \vee z<0$} (fin);
		\draw[tran] (init) to node[font=\scriptsize,draw, fill=white, 
		rectangle,pos=0.5,text width=0.8cm, text centered] {$x\geq 
		0$\\$\wedge$\\$z\geq 0$} 
		(as1);
		\draw[tran] (as1) -- node[auto,font=\scriptsize] {$y:=z$} (inner);
		\draw[tran] (inner) to node[font=\scriptsize,draw, fill=white, 
		rectangle,pos=0.5] {$y<3$} (as3);
		\draw[tran] (inner) to node[font=\scriptsize,draw, fill=white, 
		rectangle,pos=0.5] {$y\geq 3$} (as2);
		\draw[tran,rounded corners] (as2) -- (1.25,-1) -- 
		node[font=\scriptsize, 
		label={[xshift=-0.4cm,yshift=-0.2cm,font=\scriptsize] 
		{$y:=\dots$}}] 
		{} 
		(inner);
		\draw[tran, rounded corners] (as3) -- (3,0.5) -- node[font=\scriptsize, 
		label={[font=\scriptsize, yshift=-0.1cm] {$x:=x-1$}}] {} (0,0.5) -- 
		(init);
		\end{tikzpicture}
		%	\hspace{1cm}
		%	\usebox{\uniintt}
		\caption{Example for slicing illustration: program and its pCFG.}
		\label{fig:uniint4}
	\end{subfigure}
	\hspace{0.5cm}
	\begin{subfigure}{0.45\textwidth}
	\usebox{\unintthree}
	%	\hspace{1cm}
	%	\usebox{\uniintt}
	\caption{Program where the outer loop does not have a PVSM that 
		typechecks.}
	\label{fig:uniint2}
	\end{subfigure}
\caption{Program illustrations.}
\label{fig:comp-ex}
\end{figure}

Indeed, taking a closer look at typesystem in~\cite{HolgerPOPL}, there are several reasons for typechecking of PVSM to fail. The major ones are:
\begin{compactenum}
\item
A PVSM $\lem$ for \PP{} $\program$ will not typecheck if $\program$ has a 
nested loop in which the value of $\lem$ can change unboundedly in a single 
step (see Figure~\ref{fig:uniint}). 
\item
A PVSM $\lem$ for  \PP{} $\program$ will not typecheck if $\program$ has a 
nested loop which itself has a nested loop in which some variable appearing in 
some expression in $\lem$ is modified, see 
Figure~\ref{fig:uniint2}.\footnote{Both these statements regarding typechecking 
failure are somewhat simplified, even in these two cases the PVSM might 
sometimes typecheck correctly, in case where the nested loops are followed by 
assignments which completely overwrite the effect of these loops, e.g. if 
the program in Figure~\ref{fig:uniint2} contained an assignment $x:=0$. 
However, the statements intuitively 
summarize the major reasons for typechecking failure. }
\end{compactenum}



Thus, the typechecking algorithm may rule out programs where the termination-controlling variable represents e.g. a length of an array, which can be doubled/halved in some sub-program do to (de)allocation, merging, or splitting.
To overcome the rather strict typechecking, we use the results on LexRSMs to define a new notion of compositional ranking supermartingales, which we call \emph{non-negative compositional (NC)} supermartingales. For the sake of generality, we allow NC martingales to be general measurable maps, not necessarily propositionally linear.

\begin{definition}
\label{def:nonneg-comp}
A 1-dimensional measurable map $\lem$ is an NC supermartingale (NCSM) for a program $\program$ supported by an invariant map $\inv$ if there exists $\eps>0$ such that $\lem$ is:
\begin{compactenum}
\item  non-negative in each $(\loc,\vec{x})$ where $\loc$ is a location of $\pCFG_{\program}$ and $\vec{x}\in \inv(\loc)$;
\item 
 $\eps$-$\inv$-ranking in each location $\loc\in\slice{(\program)}$; and
\item 
  $\inv$-unaffecting in each $\loc\in\loops(\program)$.
\end{compactenum}
A (propositionally) linear NCSM (LinNCSM) is a NCSM which is also a (propositionally) linear expression map.
\end{definition}



We can prove that NCSMs are a sound method for proving a.s. termination in a compositional way, without any additional assumptions.

\begin{theorem}
\label{thm:nonneg-comp}
Let $\program$ be a \PP{} of the form $\textbf{while } \Psi \textbf{ do } 
\programbody \textbf{ od}$. Assume that each nested loop of $\program$ 
terminates almost surely from each reachable configuration, and that there 
exists a NCSM for $\program$ supported by some invariant map. Then $\program$ 
terminates almost surely from each initial configuration.
\end{theorem}
\begin{proof}[Proof (Key Idea)]
%Assume that there is set of non-terminating runs $A$ of positive probability.
%Let $\{X_i\}_{i=0}^{\infty}$ be a stochastic process returning the value of 
%NCSM $\lem$ in step $i$. We show that for each $i$ it holds $\E[X_i] 
%\leq \lem(\locinit,\vecinit) - \eps\cdot \E[\sharp_i]$, where $\sharp_i$ is the 
%random variable returning the number of times a location from 
%$\slice(\program)$ is visited. Since all sub-loops of $\program$ terminate 
%a.s., $\lim_{i\rightarrow \infty}\sharp_i(\run)=\infty$ for $\run\in A$. Since 
%$A$ has positive probability $\lim_{i\rightarrow 
%\infty}\E[\sharp_i(\run)=\infty]$, it holds $\lim_{i\rightarrow 
%\infty}\E[X_i] = -\infty$, a contradiction with non-negativity of $\lem$.
Let $\{X_i\}_{i=0}^{\infty}$ be a stochastic process returning the value of 
NCSM $\lem$ in step $i$. Then $\{X_i\}_{i=0}^{\infty}$ is a (non-strict) $1$-dimensional $\eps$-LexRSM for the termination time of the program, for some $\eps>0$. Since all sub-loops of $\program$ terminate, with probability one each run has level $<2$ in infinitely many steps. From theorem $\ref{thm:lexrsm-main}$ it follows that $\probm^\sigma_{\vecinit}(\ttime<\infty)=1$, for all $\sigma$ and $\vecinit$.
\end{proof}

Hence, NCSMs effectively trade the uniform integrability condition for 
non-negativity over the whole program. We believe that the latter condition is 
substantially easier to impose and check automatically, especially in the case 
of affine probabilistic programs and (propositionally) linear NCSMs: for 
\APP{}s, synthesizing NCSM entails synthesizing sufficient program invariants 
(for which there is a good automated tool support~\cite{FG10:aspic})  encoding 
the ranking, 
unaffection, and non-negativity conditions into a collection of linear 
constraints (as for general LinLexRSMs in Section~\ref{sec:algo}). 
Figures~\ref{fig:uniint} and~\ref{fig:uniint2} show instances where attempts to 
to prove of a.s. termination via PVSMs fail while proofs via LinNCSMs work. 

\begin{example}
For all programs in Figures~\ref{fig:uniint} and~\ref{fig:uniint2} there are 
LinNCSMs for all the loops in the program, which shows that the programs 
terminate 
a.s. In Figure~\ref{fig:uniint}, for both programs the inner loops have 
LinNCSMs of the form $c+d_{\loc}$, for $d_{\loc}$ a location-specific constant, 
while the outer loops have LinNCSMs of the form $x+d_{\loc}'$. In 
Figure~\ref{fig:uniint2} the program similarly has LinNCSMs defined, proceeding 
from the innermost loop and neglecting the location-specific constants, by 
variables $z$, $y$, $x$.
\end{example}
%While we defined NCSMs as one-dimensional measurable maps to make them analogous to 

Using LinNCSMs we can devise the following simple algorithm for compositional proving of almost-sure termination of \APP{}s:

\begin{algorithm}
\SetKwInOut{Input}{input}\SetKwInOut{Output}{output}
\DontPrintSemicolon

\Input{An \APP{} $\program$ together with an invariant map $\inv$.}
$d\leftarrow \text{ depth of loop nesting in $\program$}$\;
\For{$i \leftarrow d$ \KwTo $0$}{
\ForEach{sub-loop $\program'$ of $\program$ nested $i$ levels below the main loop}{
\label{algoline:one}
$\linsystem \leftarrow$ system of lin. constraints encoding the existence of LinNCSM for $\program'$ supported by $\inv$\;
\label{algoline:two}\If{$\linsystem$ not solvable}{check existence of a PVSM for $\program'$~\cite{HolgerPOPL}
\;
\lIf{PVSM does not exist}{\Return{``cannot prove a.s. termination of $\program$''}}
}
}
}
\Return{``$\program$ terminates a.s.'' }
\caption{Compositional Termination Proving}
\label{algo:compositional}
\end{algorithm}

The soundness of the algorithm follows from Theorems~\ref{thm:nonneg-comp} and 
\ref{thm:holger-comp}. Note that we can use the PVSM-based algorithm 
of~\cite{HolgerPOPL} as a back-up sub-procedure for the case when LinNCSM-based 
proof fails. Hence, Algorithm~\ref{algo:compositional} can compositionally 
prove a.s. termination of strictly larger class of programs than the PVSM-based 
algorithm alone.

To summarize, the novelty of NCSMs is the following:
\begin{compactenum}
\item
NCSMs allow compositional, fully automated proofs of a.s. termination without the need for reasoning about uniform integrability.
\item LinNCSMs are capable of proving a.s. termination of programs for which no uniformly integrable PVSMs exist (and hence the method of~\cite{HolgerPOPL} cannot be used at all on such programs).
\item LinNCSMs are capable of proving a.s. termination of programs for which the method of~\cite{HolgerPOPL} cannot be applied in an automated way, do to failure of the typechecking procedure.
\end{compactenum}

\subsection{Multidimensional Compositional Ranking}

 Above, we defined NCSMs as one-dimensional objects, to make them analogous to 
 PVSMs for better comparison. However, we can also define a multi-dimensional 
 version of NCSMs, to take advantage of the fact that LexRSMs can handle loops 
 for which no 1-dimensional linear RSM exists (see Example~\ref{ex:oneloop}). 
 We say that, given an invariant map $\inv$, an $n$-dimensional measurable map 
 is $\eps$-$\inv$-ranking in a location $\loc$ if for each $\vec{x}\in 
 \inv(\loc)$ and each gen. transition $\tilde{\tau}$ outgoing from $\loc$ there 
 exists $1\leq j \leq n$ such that $\tilde\tau$ is $\eps$-ranked by $\lem_j$ 
 and for each $j'<j$ we have that  $\tilde{\tau}$ is unaffected by $\lem_{j'}$.

\begin{definition}
An $n$-dimensional measurable map $\overrightarrow{\lem}=(\lem_1,\dots,\lem_n)$ 
is an NC supermartingale (NCSM) for a program $\program$ supported by an 
invariant map $\inv$ if there exists $\eps>0$ such that:
\begin{compactenum}
	\item  for each $1\leq i \leq n$, $\lem_i$ is non-negative in each location of $\pCFG_{\program}$;
	\item 
	$\overrightarrow{\lem}$ is $\eps$-$\inv$-ranking in each location 
	$\loc\in\slice{(\program)}$; and
	\item 
	for each $1\leq i \leq n$, $\lem_i$ is
	unaffected in each $\loc\in\loops(\program)$.
\end{compactenum}
An $n$-dimensional linear NCSM (LinNCSM) is an $n$-dimensional NCSM which is also a linear expression map.
\end{definition}

The following theorem can be proved in essentially the same way as Theorem~\ref{thm:nonneg-comp}.

\begin{theorem}
Let $\program$ be a \PP{} of the form $\textbf{while } \Psi \textbf{ do } 
\programbody \textbf{ od}$. Assume that each nested loop of $\program$ terminates almost surely from each reachable configuration, and that there exists a NCSM for $\program$ supported by some invariant map. Then $\program$ terminates almost surely.
\end{theorem}

For \APP{}s, we can generalize Algorithm~\ref{algo:compositional} by changing line~\ref{algoline:one} to ``check existence of a multi-dimensional LinNCSM for $\program'$'' and line~\ref{algoline:two} to ``\textbf{if} a multi-dimensional LinNCSM does not exist.'' The check of existence of a multi-dimensional LinNCSM for $\program'$ can be done by algorithm presented in Section~\ref{sec:algo}, modified so as to only pursue ranking for generalized transitions outgoing from locations belonging to $\slice(\program')$. (I.e., only these gen. transitions have $\eps_{\tau}$ included in the objective function, and algorithm terminates once all such transitions are ranked.) %This further expands the usage of the algorithm.

%There are several advantages of using LexRSMs in a compositonal way. First, using Algorithm~\ref{algo:compositional} (modified for multi-dimensional LinNCSMs) we opt for solving a larger number of smaller linear systems rather than a smaller number of larger ones, which may be sometimes advantageous for performance reasons. Second, using compositional reasoning, we can leverage LexRSMs in a situation where some of the sub-loops were already proved to be a.s. terminating by some other argument, such as using an interactive proof assistant that leverages human intuition. This makes the LexRSM method applicable to programs whose parts are difficult to treat via automated static analysis.


%% I SHOULD MAYBE EXPLAIN THAT THE NEED TO PERFORM A STATIC ANALYSIS OF THE WHOLE PROGRAM IS 
%% NOT LIMITING, AS SOMETHING SIMILAR NEEDS TO BE DONE FOR PVSM. BUT MAYBE LEAVE THIS FOR REBUTTAL. :)

%Each measurable map for $\lem$ can be, in a natural way, restricted to a 
%measurable map $\lem_{\program'}$ for $\program'\in \loops(\program)$, 
%since each pCFG $\pCFG_{\program'}$ is a sub-pCFG of $\pCFG_{\program}$. 
%Similarly, a measurable map $\lem$ for $\program$ can be restricted to a 
%measurable map $\lem_{\slice}$ for $\slice(\program)$, since 
%$\pCFG_{\slice(\program)}$ is produced from $\pCFG_{\program}$ essentially by 
%removing certain locations (provided that we identify the locations forming 
%heads of nested loops with locations preceding skip statements), and 
%$\lem_{\slice}$ is defined to be equal to $\lem$ in all the locations that 
%remained. Similarly, 

