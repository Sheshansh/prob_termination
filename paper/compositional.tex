\section{Compositionality of Ranking Supermartingales Revisited}
\label{sec:compositional}

Compositionality in the context of termination proving means providing the 
proof of termination step-by-step, handling one loop at a time, rather than 
attempting to construct the proof (in our case, a LexRSM) at once, by solving a 
single large system of constraints. In non-probabilistic setting, the 
relationship between lexicographic ranking functions and compositional 
termination proving is well established, as well as the importance of 
compositional proofs for efficient implementation~\cite{xxx}. In the context of 
probabilistic programs, the work~\cite{HolgerPOPL} attempted to provide a 
compositional notion of almost-sure termination proof based on the 
\emph{probabilistic variant rule (V-rule),} which we explain in a more detail 
below. 
However, as already noted in~\cite{HolgerPOPL}, the probabilistic V-rule is not 
sound in general. In order to rectify this issue, in~\cite{HolgerPOPL} they 
impose an additional constraint of \emph{uniform integrability} on ranking 
supermartingales that prove termination of individual loops, 
under which the probabilistic V-rule is shown to be sound. Apart from uniform 
integrability being somewhat restrictive in itself, in~\cite{HolgerPOPL} it is 
argued that proving uniform integrability is beyond the capability of 
state-of-the-art automated theorem provers. As a substitute for 
these,~\cite{HolgerPOPL} introduces a type system that can be used to 
automatically prove 
uniform integrability of ranking supermartingales for a restricted class of 
programs. In particular, the method cannot handle programs in which termination 
is controlled by variable that can be modified by a non-constant value, e.g. 
variable $x$ which appears in an assignment of the form $x:=x-y$. In this 
section we show that using our insights in lexicographic supermartingales we 
can obtain a subtly different notion of a probabilistic V-rule which is sound 
without any additional assumptions, and which can be used to prove termination 
of programs that the previous method cannot handle. 