%\vspace{-1em}
\section{Invariants and Ranking Supermartingales}\label{sec:invm}
%\vspace{-0.5em}

In this section we recall basic results on (one-dimensional) ranking 
supermartingales in probabilistic programs.

%In this section we recall known methods and constructs for solving the 
%qualitative termination and reachability questions for \APP{s}, namely  
%linear invariants and ranking supermartingales. 
%We also demonstrate that these methods are not sufficient to address the 
%quantitative variants of these questions (i.e., probabilistic termination).
%In order to discuss the necessary concepts, we recall the basics of 
%martingales, which is relevant for both this and subsequent sections. 
%Since the definitions of this section are from the literature we relegate
%relevant illustrations through example to the Appendix.

%While the core results of this section were already proven in the previous 
%literature, the version we sometimes present slight generalizations of these 
%result, which make them applicable to a larger class of programs. We discuss 
%the nature of these generalizations where appropriate.

%\vspace{-1em}
\subsection{Invariants}
%\vspace{-0.5em}

Invariants are a vital element of many program analysis techniques. 
Intuitively, invariants are maps assigning to each 
program location $\loc$ of some pCFG a predicate which is guaranteed to hold 
whenever $\loc$ is 
entered. 

\smallskip
\begin{definition}[Linear Predicate Maps (LPMs) and Invariants] We define the 
following:
\begin{compactenum}
\item
A \emph{linear predicate map (LPM)} for an \APP{} $P$ is a function $I$ 
assigning to each location $\loc$ of the pCFG $\pCFG_P$ a propositionally 
linear predicate $I(\loc)$ over the set of program variables of $P$.
\item
A \emph{linear invariant} (or just an invariant) for an \APP{} $P$ is 
a linear predicate map $\inv$ for $P$ with
the following property: for each location $\loc$ of $\pCFG_P$ and each finite 
path $(\loc_0,\vec{x}_0),\cdots,(\loc_n,\vec{x}_n)$ such that 
$(\loc_0,\vec{x}_0)=(\locinit,\vecinit)$ and $\loc_n = \loc$ it holds
$\vec{x}_n\models \inv(\loc)$.
\end{compactenum}
%%A 
%%probabilistic LPM for $P$ is a tuple $(\pinv,\pinfunc)$ such that $\pinv$ is 
%%a 
%%pure LPM for $P$ and $\pinfunc$ is a function assigning to each location 
%%$\loc$ 
%%a probability $\pi(\loc)\in[0,1]$.
\end{definition}



%A special class of LPMs are program invariants, i.e. maps assigning to each 
%program location $\loc$ an PLP which is guaranteed to hold whenever $\loc$ is 
%entered.
%Invariants are a standard construct used in automated program analysis. 
%To avoid confusion with probabilistic invariants, we call the 
%standard invariants \emph{pure}.
%
%\begin{definition}[Pure Linear Invariant]
%A \emph{pure linear invariant} (or just a pure invariant) for an \APP{} $P$ is 
%%a function map $\inv$  assigning to each location $\loc$ of the pCFG 
%%$\pCFG_P$ 
%%a propositionally 
%%linear predicate $I(\loc)$ over the set of program variables of $P$ 
%a linear predicate map $\inv$ for $P$ with
%the following property: for each location $\loc$ of $\pCFG_P$ and each finite 
%path $(\loc_0,\vec{x}_0),\cdots,(\loc_n,\vec{x}_n)$ such that 
%$(\loc_0,\vec{x}_0)=(\locinit,\vecinit)$ and $\loc_n = \loc$ it holds
%$\val_{\vec{x}_i}\models \inv(\loc)$.
%\end{definition}



%\vspace{-1em}
\subsection{Supermartingales}
%\vspace{-0.5em}

%Submartingales and their counterparts, supermartingales, are a standard tool 
%of 
(Super)martingales, are a standard tool of 
probability theory apt for analyzing probabilistic objects arising in computer 
science, from automata-based models~\cite{BKKNK:pMC-zero-reachability} to 
general probabilistic 
programs~\cite{SriramCAV,HolgerPOPL,CFNH16:prob-termination,CFG16:positivstellensatz-arxiv,BEFH16:doob-decomposition}.
 
%\PN{Maybe move this sentence to intro.} 
Let us first recall basic
definitions and results related to supermartingales, which we need in our 
analysis.

\smallskip\noindent{\bf Conditional Expectation.} 
%The notion of conditional expectation 
%tightly connected to martingales. We present only a brief informal exposition 
%here, for formal treatment see, e.g.~\cite[Chapter 9]{Williams:book}.
Let $(\Omega,\mathcal{F},\probm)$ be a probability space, 
$X\colon\Omega\rightarrow 
\Rset$ an $\mathcal{F}$-measurable function, and $\mathcal{F}'\subseteq 
\mathcal{F}$ sub-sigma-algebra of $\mathcal{F}$. The \emph{conditional 
expectation} of $X$ given $\mathcal{F}'$ is an $\mathcal{F}'$-measurable random 
variable denoted by $\E[X| \mathcal{F}']$ which satisfies, for each set $A\in 
\mathcal{F}'$, the following: 
\begin{equation}
\label{eq:cond-exp}
\E[X\cdot 1_A] = \E[\E[X|\mathcal{F}]\cdot 1_A],
\end{equation}
where $1_A \colon \Omega\rightarrow \{0,1\}$ is an \emph{indicator function} of 
$A$, i.e. function returning $1$ for 
each $\omega\in A$ and $0$ for each $\omega\in \Omega\setminus A$. Note that 
the left hand-side of~\eqref{eq:cond-exp} intuitively represents the expected 
value 
of $X(\omega)$ with domain restricted to $A$.

Recall that in context of probabilistic programs we work with probability 
spaces of the form $(\Omega,\natfilt,\probm^\sigma)$, where $\Omega$ is a 
set of runs in some $\pCFG$ and $\natfilt$ is (the smallest) sigma-algebra 
such that all the functions $\cfg{\sigma}{i}$, where $i\in \Nset_0$ and 
$\sigma$ is a scheduler, are $\natfilt$-measurable. In such a setting we can 
also consider sub-sigma-algebras $\natfilt_i$, $i\in \Nset_0$, of 
$\natfilt$, where $\natfilt_i$ is the smallest sub-sigma-algebra of 
$\natfilt$ such that all the functions $\cfg{\sigma}{j}$, $0\leq j \leq 
i$, are $\natfilt_i$-measurable. Intuitively, each set $A$ belonging to such 
an $\natfilt_i$ consists of runs whose first $i$ steps satisfy some 
property, and the probability space $(\Omega,\natfilt_i,\probm^\sigma)$ 
allows us to reason about probabilities of certain events happening in the 
first 
$i$ steps of program execution. 
%(note that since $\natfilt_i \subseteq 
%\natfilt$, we can view $\probm^\sigma$ also as a function of type 
%$\natfilt_i\rightarrow [0,1]$). 
Then, for each $A\in \natfilt_i$, the 
value $\E[\E[X|\natfilt_i]\cdot 1_A]$ represents the expected value of 
$X(\run)$ for the randomly generated run $\run$ provided that we restrict to 
runs whose
prefix of length $i$ satisfies the property given by $A$. 
Note that the sequence $\natfilt_0,\natfilt_1,\natfilt_2,\dots$ forms a 
filtration of $\natfilt$, which we call a \emph{canonical filtration}.

%%\paragraph*{Basics of Supermartingales}

\smallskip
\begin{definition}[Supermartingale]
Let $(\Omega,\mathcal{F},\probm)$ be a probability space and 
$\{\mathcal{F}_i\}_{i=0}^{\infty}$ a filtration of $\mathcal{F}$. A sequence of 
random variables $\{X_i\}_{i=0}^{\infty}$ is a \emph{supermartingale} w.r.t. 
filtration $\{\mathcal{F}_i\}_{i=0}^{\infty}$ if it satisfies these conditions:
\begin{compactenum}
\item  The process $\{X_i\}_{i=0}^{\infty}$ is \emph{adapted} to 
$\{\mathcal{F}_i\}_{i=0}^{\infty}$, i.e. for all $i\in \Nset_0$ it holds that 
$X_i$ is $\mathcal{F}_i$-measurable.
\item For all $i\in \Nset_0$ it holds $\E[|X_i|]<\infty$.
\item For all $i\in \Nset_0$ it holds 
\begin{equation}
\label{eq:supermart-def}
\E[X_{i+1}|\mathcal{F}_i] \leq X_i.
\end{equation}

A supermartingale $\{X_i\}_{i=0}^{\infty}$ has $c$-bounded differences, where 
$c\geq 0$, if $|X_{i+1}-X_i|<c$ for all $i\in \Nset_0$
\end{compactenum}
%A sequence of 
%random variables $\{X_i\}_{i=0}^{\infty}$ is a \emph{submartingale} w.r.t.  
%$\{\mathcal{F}_i\}_{i=0}^{\infty}$ if the process $\{-X_i\}_{i=0}^{\infty}$ is 
%a supermartingale w.r.t. the same filtration.
\end{definition}


Intuitively, a supermartingale is a stochastic process whose average value is 
guaranteed not to rise as time evolves, even if some information on the past 
evolution of the process is revealed. 
%Symmetrically, a submartingale is a 
%process whose value does not drop on average. 
We often need to work with supermartingales whose value 
is guaranteed to decrease on average, until a certain condition is 
satisfied. The point in time in which such a condition is satisfied is called a 
\emph{stopping time}.

\smallskip
\begin{definition}[Stopping time]
Let $(\Omega,\mathcal{F},\probm)$ be a probability space and 
$\{\mathcal{F}_i\}_{i=0}^{\infty}$ a filtration. A random variable $\stime 
\colon 
\Omega\rightarrow \Nset_0$ is called a \emph{stopping time} w.r.t. 
$\{\mathcal{F}_i\}_{i=0}^{\infty}$ if %$\stime$ is adapted to this filtration 
%and if 
for all $j\in \Nset_0$ the set $\{\omega\in \Omega\mid \stime(\omega)\leq j\}$ 
belongs to $\mathcal{F}_j$.
\end{definition}


In particular, for each set of configurations $C$ the reachability time 
$\treach{C}$ of $C$ is a stopping time w.r.t. the canonical filtration, since 
at each time $j$ we can decide whether $\treach{C}>j$ or not by looking at the 
prefix of a run of length $j$. 
%When $\Omega$ is a set of runs of some $\pCFG_P$, we can think of 
%$\stime$ as representing a 
%certain condition which can or cannot hold in a given time instance of the 
%execution of $P$, where $\stime(\run)$ represents the first point in time 
%during the execution represented by $\run$ in which $P$ holds. The condition 
%$\{\omega\in \Omega\mid \stime(\omega)\leq j\}\in \mathcal{F}_j$ then states 
%that whether $P$ holds at step $j$ of a run can be determined by looking at 
%the 
%prefix of the run of length $j$. In other words, the stopping condition cannot 
%``talk'' about future events. The typical stopping time we use in the context 
%of \APP{s} is the \emph{reachability} time $\treach{C}$ for some set of 
%configurations $C$, where $\treach{C}(\run)$ is the first point in time when 
%the current configuration on $\run$ is in $C$. In particular, if $C$ is the 
%set 
%of all configurations $(\loc,\vec{x})$ such that $\loc=\loc^\lout_P$ (the 
%terminal location of $\pCFG_P$), then we denote $\treach{C}$ by $\ttime$ and 
%call it a \emph{termination time} of $P$.
%We combine stopping conditions with martingales to get 
%\emph{$\eps$-decreasing} 
%supermartingales. %
Finally, we recall the fundamental notion of a ranking 
supermartingale.

\smallskip
\begin{definition}[Ranking supermartingale]
\label{def:ranking}
Let $(\Omega,\mathcal{F},\probm)$ be a probability space, 
$\{\mathcal{F}_i\}_{i=0}^{\infty}$ a filtration of $\mathcal{F}$, $\stime$ a 
stopping time w.r.t. that filtration, and $\eps\geq 
0$. 
A supermartingale $\{X_i\}_{i=0}^{\infty}$ (w.r.t. 
$\{\mathcal{F}_i\}_{i=0}^{\infty}$) is \emph{$\eps$-decreasing} until 
$\stime$ 
if it 
satisfies 
the following additional condition: for all $i\in \Nset_0$ it holds
\begin{equation}
\label{eq:ranking-sup}
\E[X_{i+1}|\mathcal{F}_i] \leq X_i - \eps\cdot\vec{1}_{\stime > i}.
\end{equation}

Further, $\{X_i\}_{i=0}^{\infty}$ is an \emph{$\eps$-ranking} supermartingale 
($\eps$-RSM) for $\stime$ if it 
is $\eps$-decreasing until $\stime$ and for 
each $\omega\in\Omega, j\in\Nset_0$ it holds $\stime(\omega)>j \Rightarrow 
X_j(\omega)\geq 0$.
%is a supermartingale w.r.t. $\{\mathcal{F}_i\}_{i=0}^{\infty}$. 
%A supermartingale $\{X_i\}_{i=0}^{\infty}$ is 
%is $\eps$-repulsing if for all $i\in \Nset_0$ it holds
%\begin{equation}
%\label{eq:repulsing-sub}
%\E[X_{i+1}|\mathcal{F}_i] \leq X_i - \eps\cdot\vec{1}_{\stime > i}.
%\end{equation}
\end{definition}
Intuitively, if $\stime$ is the reachability time $\treach{C}$ of some set $C$, 
then the previous definition requires that an $\eps$-ranking supermartingale 
must decrease by at least $\eps$ on average up to the point when $C$ is 
reached for a first time. After that, it must not increase (on average).
The above definition is a bit more general than the standard one in the
literature as we also consider reachability as opposed to only termination.


%Intuitively, until the condition specified by $\stime$ is satisfied, ranking 
%supermartingale is non-negative and at the same time decreasing by at least 
%$\eps$ on average, forming a probabilistic analogue of the classical ranking 
%functions. 

%We finish this part by noting that for some applications it is preferable to 
%work with supermartingales that have \emph{bounded differences.}

%\begin{definition}
%A supermartingale $\{X_i\}_{i=0}^{\infty}$ has $c$-bounded differences, where 
%$c\geq 0$, if $|X_{i+1}-X_i|<c$ for all $i\in \Nset_0$.
%\end{definition}
%Note that $\{X_i\}_{i=0}^{\infty}$ being an $\eps$-repulsing supermartingale 
%for 
%$\stime$ is 
%\emph{not} equivalent to $\{-X_i\}_{i=0}^{\infty}$ being an $\eps$-ranking 
%supermartingale for $\stime$.

%We typically work with $\eps$-ranking supermartingales whose initial value is 
%positive and $\eps$-repulsing supermartingales whose initial value is 
%negative. 
%Then, intuitively an $\eps$-ranking supermartingale is driven towards negative 
%values by at least $\eps$ in each step (on average) until it indeed becomes 
%negative, while an $\eps$-repulsing submartingale is driven \emph{away} from 
%positive values by at least $\eps$ per step, as long as it stays non-positive 
%(we cannot rule out that it eventually becomes positive, as the decrease by 
%$\eps$ is guaranteed only on average). 
%%Note that $\{X_i\}_{i=0}^{\infty}$ being an $\eps$-repulsing submartingale is 
%%\emph{not} equivalent to $\{-X_i\}_{i=0}^{\infty}$ being an $\eps$-ranking 
%%supermartingale.

\smallskip\noindent{\bf Martingales in Program Analysis.}
In the context of \APP{} analysis, we consider a special type of 
supermartingales given as functions of the current values of program variables. 
In this paper we focus on the case when these functions are \emph{linear}.

\smallskip
\begin{definition}[Linear Expression Map]
A \emph{linear expression map (LEM)} for an \APP{} $P$ is a function $\lem$ 
assigning to each program location $\loc$ of $\pCFG_P$ an affine expression 
$\lem(\ell)$ over the program variables of $P$.
\end{definition}

Each LEM $\lem$ and location $\loc$ determines an affine function $\lem(\loc)$ 
which takes as an argument an $n$-dimensional vector, where $n$ is the number 
of distinct variables in $P$. We use $\lem(\loc,\vec{x})$ as a shorthand 
notation for $\lem(\loc)(\vec{x})$. 
Martingales for \APP{} analysis are defined via a standard notion of 
pre-expectation~\cite{SriramCAV}. Intuitively, a pre-expectation of $\lem$ is a 
function which for each configuration $(\loc,\vec{x})$ returns the maximal
expected value of $\lem$ after one step is made from this configuration, where 
the maximum is taken over all possible non-deterministic choices.

\smallskip
\begin{definition}[Pre-Expectation]
Let $P$ be an \APP{} such that $\pCFG_P = 
(\locs,\pvars,\locinit,\vecinit,\transitions,\probdist,\guards)$ and let $\lem$ 
a 
linear expression map 
for~$P$. 
%Let $\locs$ 
%be the set of locations of $\pCFG_P$ and $\pvars$ its set of program 
%variables. 
The 
pre-expectation of $\lem$ is a function $\preexp{\lem}\colon \locs\times 
\Rset^{|\pvars|} \rightarrow \Rset$ defined as follows:
\begin{compactitem} %\itemsep1pt \parskip0pt \parsep0pt
\item 
if $\loc$ is a probabilistic location, then
$$\preexp{\lem}(\loc,\vec{x}):=\sum_{(\loc,1,\id_1,\loc')\in\transitions} 
Pr_{\loc}\left((\loc,1,\id_1,\loc')\right)\cdot
 \lem(\loc',\vec{x});$$
\item 
 if $\loc$ is a non-deterministic location, then
$$\preexp{\lem}(\loc,\vec{x}):=
\max_{(\loc,1,\id_1,\loc')\in\transitions}\lem(\loc',\vec{x});$$

%\item 
%$\mathrm{pre}_\eta(\loc,\mathbf{x}):=\min_{(\loc,id,\ell')\in\transitions}\eta\left(\loc',\mathbf{x}\right)$
% if $\loc$ is an angelic location;
\item 
if $\loc$ is a deterministic location, then for each $\vec{x}$ the value 
$\preexp{\lem}(\loc,\vec{x})$ is determined as follows: there is exactly one 
transition
$\tau=(\loc,j,\up,\loc')$ such that $\vec{x}\models G(\tau)$. We distinguish 
three cases:
\begin{compactitem}
\item If $\up\colon \Rset^{|\pvars|}\rightarrow \Rset$ is a function, then 
$$\preexp{\lem}(\loc,\vec{x}):= \lem(\loc',\vec{x}(j\leftarrow \up(\vec{x}))).$$
\item If $\up$ is a distribution $d$, then $$ \preexp{\lem}(\loc,\vec{x}):= 
\lem(\loc',\vec{x}(j\leftarrow \E[d])),$$ where $\E[d]$ is the expected value of the 
distribution $d$.
\item 
If $\up$ is a set, then $$ \preexp{\lem}(\loc,\vec{x}):= \max_{a\in\up}
\lem(\loc',\vec{x}(j \leftarrow a)).$$
\end{compactitem}
%$\mathrm{pre}_\eta(\loc,\mathbf{x}):=\eta(\loc',\expv_{\rvars}\left(f(\vec{x},\mathbf{r})\right))$
% if $\loc$ is a deterministic location, $(\loc,f,\loc')\in\transitions$ and 
%$\mathbf{x}\in G(\loc,f,\loc')$, where 
%$\expv_{\rvars}\left(f(\vec{x},\mathbf{r})\right)$ is the expected value of 
%$f(\vec{x},\cdot)$. %on $\rvars$ if $\vec{x}$ is treated as a constant vector.
\end{compactitem}
%\end{definition}
\end{definition}

\smallskip
\begin{definition}(Linear Ranking Supermartingale)
\label{def:lrsm}
Let $P$ be an \APP{} such that $\pCFG_P = 
(\locs,\pvars,\locinit,\vecinit,\transitions,\probdist,\guards)$, let $\inv$ be 
a linear predicate map and let 
$\confset\subseteq \locs \times \Rset^{|\pvars|}$ be some set of 
configurations. 
%\begin{compactenum}
%\item
A linear $\eps$-ranking supermartingale ($\eps$-LRSM) for $\confset$ 
supported by $\inv$ is a 
linear expression map $\lem$ for $P$ such that for all configurations 
$(\loc,\vec{x})$ of $\pCFG_P$ with
$(\loc,\vec{x})\not\in \confset$ and $\vec{x}\models \inv(\loc)$ the following 
two conditions hold:
\begin{compactitem}
%\item For all $(\loc,\vec{x})\in \locs\times\Rset^{|\pvars|}$ it holds
%\item For all  it holds 
\item
$\lem(\loc,\vec{x})\geq 0$
\item 
$\preexp{\lem}(\loc,\vec{x}) \leq \lem(\loc,\vec{x})-\eps$ 
\end{compactitem}
%\item
%\end{compactenum}
A linear $\eps$-ranking supermartingale supported by $\inv$ has 
$c$-bounded differences if for each $(\loc,\vec{x})$ such that $\vec{x}\models 
\inv(\loc)$ and each configuration $(\loc',\vec{x}')$ such that $(\loc,\vec{x}) 
(\loc',\vec{x}')$ is a path in $\pCFG_P$ it holds 
$|\lem(\loc,\vec{x})-\lem(\loc',\vec{x}')|\leq c$. 
\end{definition}


The relationship between $\eps$-LRSM in \APP{s}, (pure) invariants, and almost-sure termination 
is summarized in the following theorem. 
\smallskip
\begin{theorem}[{\cite[Theorem 1]{CFNH16:prob-termination}}]
\label{thm:old-ranking}
Let $P$ be an \APP{}, $\sigma$ a scheduler, and 
$(\Omega,\natfilt,\probm^\sigma)$ the corresponding probability space. Further, 
let $C$ be the set of terminating configurations of $\pCFG_P$ (i.e., the termination
location is reached), such that there exist an $\eps>0$ and an $\eps$-linear ranking 
supermartingale $\lem$ supported by a pure invariant $I$. 
%%for $\treach{C}$. 
Then 
\begin{compactenum}
\item
$\probm^{\sigma}(\ttime<\infty)=1$, i.e. termination is ensured almost-surely.
\item
$\E^{\sigma}[\ttime]<\lem(\locinit,\vecinit)/\eps$.
%\item
%If $\{X_i\}_{i=0}^{\infty}$  has $c$-bounded differences for some $c$, then 
%there exists a concentration bound $B$ for $\treach{C}$ such that $B\leq 
%({E^{\sigma}[X_0]}/{\eps})+1$.
\end{compactenum}
\end{theorem}


The previous result shows that if there exists  an $\eps$-LRSM 
supported by a pure invariant $I$, for $\eps>0$, then under each scheduler 
termination is ensured almost-surely.
We now demonstrate that pure invariants, though effective for almost-sure 
termination, are ineffective to answer probabilistic termination questions.


%\begin{figure}[t]
\lstset{language=affprob}
\lstset{tabsize=3}
\newsavebox{\nonterm}
\begin{lrbox}{\nonterm}
\begin{lstlisting}[mathescape]
$x:=30,y:=20$
while $y\geq 0$ do
	$x:=x+$sample$(\mathrm{Uniform}[-\frac{1}{4},1])$
	$y:=y+$sample$(\mathrm{Uniform}[-1,\frac{1}{4}])$
	while $x \leq 0$ do skip od
od
\end{lstlisting}
\end{lrbox}
\begin{figure}[t]
%%\centering
\usebox{\nonterm}
\begin{tikzpicture}[x = 1.7cm]
%\node[det] (det1) at (0,0) {$\loc_0$};
%\node[det] (while) at (1.5,-1.5)  {$\loc_1$};
%\node[det] (fin) at (0,-3) {$\loc_5$};
%\node[ran] (prob) at (3,-1.5) {$\loc_2$};
%\node[ang] (angel) at (3,0) {$\loc_3$};
%\node[dem] (demon) at (3,-3)  {$\loc_4$};
%
%\draw[tran] (det1) to node[auto, swap] {x:=0} (while);
%\draw[tran] (while) to node[font=\scriptsize,draw, fill=white, 
%rectangle,pos=0.3] {$x<0$} (fin);
%\draw[tran, loop, looseness = 5, in =-65, out = -115] (fin) to (fin);
%\draw[tran] (while) to node[font=\scriptsize,draw, fill=white, 
%rectangle,pos=0.3] {$x\geq 0$} (prob);
%\draw[tran] (prob) to node[font=\scriptsize,draw, fill=white, 
%rectangle,pos=0.3, inner sep = 1pt] {$\frac{6}{10}$} (angel);
%\draw[tran] (prob) to node[font=\scriptsize,draw, fill=white, 
%rectangle,pos=0.3, inner sep = 1pt] {$\frac{4}{10}$} (demon);
%\draw[tran] (demon) -- node[auto] {x:=x+1} (demon-|while) -- (while);
%\draw[tran] (angel) -- node[auto, swap] {x:=x+1} (angel-|while) -- (while);
%\draw[tran] (demon) to node[auto] {x:=x-1} (while);
%\draw[tran] (angel) to node[auto, swap] {x:=x-1} (while);

\node[det] (while) at (1.5,0)  {$\loc_0$};
%%\node[det] (fin) at (0,-3) {};
\node[det] (p1) at (2.5,0) {$\loc_1$};
\node[det] (p2) at (3.5,0) {$\loc_2$};

\node[det] (lp) at (4.5,0) {$\loc_3$};
%\node[ang] (angel) at (3,0) {$\loc_3$};
%\node[dem] at (3,-3) (demon) {$\loc_4$};
%\node[det] (aif) at (2,-0.075) {$\loc_5$};
%\node[det] (aelse) at (3,-0.8) {$\loc_6$};
%\node[det] (dif) at (2,-3) {$\loc_7$};
%\node[det] (delse) at (3,-2.2) {$\loc_8$};
%\node[det] (det1) at (0,0) {$\loc_0$};
%\node[det] (while) at (1.5,-1.5)  {};
\node[det] (fin) at (0,0) {$\loc_4$};
%\node[ran] (prob) at (3,-1.5) {};
%\node[ang] (angel) at (3,0) {};
%\node[dem] at (3,-3) (demon) {};
%
%%\draw[tran] (det1) to node[auto, swap] {x:=0} (while);
\node (dum1) at (0,0.8) {};
\node (dum2) at (0,-0.8) {};
\draw[tran] (while) to node[font=\scriptsize,draw, fill=white, 
rectangle,pos=0.5] {$y<0$} (fin);
\draw[tran, loop, looseness = 5, in =-65, out = -115] (fin) to (fin);
\draw[tran] (while) -- node[font=\scriptsize,draw, fill=white, 
rectangle,pos=0.5, inner sep = 2pt] {$y\geq0$} node[label={[label 
distance=0.12cm]-90:{x:=...}}] {} (p1);
\draw[tran] (p1) to node[label={[label 
distance=0.12cm]-90:{y:=...}}] {} (p2);
\draw[tran] (lp) to  (p2);
\draw[tran] (p2) -- (p2|-dum2) -- node[font=\scriptsize,draw, fill=white, 
rectangle,pos=0.5, inner sep = 2pt] {$x\leq 0$} (lp|-dum2) -- (lp);
\draw[tran] (p2) -- (p2|-dum1) -- node[font=\scriptsize,draw, fill=white, 
rectangle,pos=0.5, inner sep = 2pt] {$x> 0$} (while|-dum1) -- (while);
%\draw[tran] (prob) to node[font=\scriptsize,draw, fill=white, 
%rectangle,pos=0.3, inner sep = 1pt] {$\frac{4}{10}$} (demon);
%\draw[tran] (demon) -- node[auto] {x:=x+1} (demon-|while) -- (while);
%\draw[tran] (angel) -- node[auto, swap] {x:=x+1} (angel-|while) -- (while);
%\draw[tran] (demon) to node[auto] {x:=x-1} (while);
%\draw[tran] (angel) to node[auto, swap] {x:=x-1} (while);


%\draw[tran] (angel) -- (aif);
%\draw[tran] (angel) -- (aelse);
%\draw[tran] (demon) -- (dif);
%\draw[tran] (demon) -- (delse);



%\draw[tran] (prob) -- node[font=\scriptsize,draw, fill=white, 
%rectangle,pos=0.5, inner sep = 1pt] {$\frac{3}{4}$} (prob|-dum1) -- node[auto] 
%{x:=x-1}
%(while|-dum1)--(while);
%\draw[tran] (prob) -- node[font=\scriptsize,draw, fill=white, 
%rectangle,pos=0.5, inner sep = 1pt] {$\frac{1}{4}$} (prob|-dum2) 
%-- node[auto,swap] {x:=x+1} (while|-dum2)--(while);
%\draw[tran] (dif) --  (demon-|while) -- node[auto, pos=0.2] {x:=x+1} (while);
%\draw[tran] (aif) --  (angel-|while) -- node[auto, swap, pos=0.2] {x:=x+1} 
%(while);
%\draw[tran] (delse) -- node[auto] {x:=x-1} (delse-|while.305) --  (while.-55);
%\draw[tran] (aelse) -- node[auto, swap] {x:=x-1} (aelse-|while.55) --  
%(while.55);
\end{tikzpicture}
\caption{A program with infinitely many reachable configurations which 
terminates with high probability, but not almost 
surely, together with a sketch of its pCFG. }\label{fig:nonterm}
\end{figure}

%We now show why this approach does not work for the probabilistic reachability and termination. 

\smallskip
\begin{example}\label{ex:nonterm}
Consider the program in Figure~\ref{fig:nonterm}. 
In each 
iteration of the outer loop each of the variables is randomly modified by 
adding a number drawn from some uniform distribution.
%incremented or 
%decremented. The probability of incrementing $x$ is $\frac{3}{8}$, for
%decrementing $x$ it is $\frac{1}{8}$; the probability of incrementing $y$ it 
%is 
%$\frac{1}{8}$ and for decrementing $y$ it is $\frac{3}{8}$. 
Average increase of $x$ in each iteration is $\frac{3}{8}$, while average decrease 
of $y$ is $-\frac{3}{8}$.  It is easy to see that a program does not 
terminate almost-surely: 
there is for instance a tiny but non-zero probability of $x$ being decremented 
by at least $\frac{1}{8}$ in 
each of the first 240 loop iterations, after which we are stuck in the infinite 
inner 
loop. On 
the other hand, the expectations above show that there is a ``trend'' of $y$ 
decreasing and $x$ increasing, and executions that follow this trend eventually 
decrement $y$ below $0$ without entering the inner loop. Hence, the 
probabilistic intuition tells us that the program terminates with a high 
probability. However, the techniques of this section cannot prove 
this high-probability termination, since existence of an $\eps$-LRSM (with 
$\eps>0$) supported by a pure invariant already implies a.s. termination,
and so no such $\eps$-LRSM can exist for the program. 
\end{example}

In the next section we generalize the notion of pure invariants to stochastic invariants
for probabilistic termination to resolve issues like Example~\ref{ex:nonterm}.

%Adirect consequence of Theorems~\ref{thm:pure-supermart} is the following 
%connection 
%between pure invariants, 
%linear ranking supermartingales, and reachability properties. 
%
%\begin{corollary}
%\label{col:linear-ranking-old}
%Let $\lem$ be a linear $\eps$-ranking supermartingale for some set $C$ of 
%configurations of an \APP{} $P$ such that $\lem$ is supported by some pure 
%invariant of $P$. Then under every scheduler $\sigma$ it holds:
%
%\begin{compactenum}
%\item
%$\probm^{\sigma}(\treach{C}<\infty)=1$, i.e. the set $C$ is reached 
%almost-surely
%\item
%$\E^{\sigma}[\treach{C}]<\lem(\locinit,\vecinit)/\eps$.
%\item
%If $\{\lem(\cfg{\sigma}{i})\}_{i=0}^{\infty}$  has $c$-bounded differences for 
%some $c$, then 
%there exists a concentration bound $B$ for $\treach{C}$ such that $B\leq 
%({\lem(\locinit,\vecinit)}/{\eps})+1$.
%\end{compactenum}
%\end{corollary}







%\begin{proof}
%Denote $X_i = \lem(\cfg{\sigma}{i})$. To prove that 
%\end{proof}

%\begin{example}
%If we apply Theorem~\ref{thm:pure-supermart} on the $\frac{1}{4}$-LRSM $\lem$ 
%in Example~\ref{ex:lrsm}, we get exactly the process $\{X'_i\}_{i=0}^{\infty}$ 
%described in Example~\ref{ex:rsm}.
%\end{example}

%Having a supermartingale which decreases, on average, by some constant until 
%$C$ is reached effectively proves that $C$ is reached almost surely, very much like 
%the standard ranking functions prove that non-probabilistic program terminates. 
%Moreover, we can use such a supermartingale to prove additional assertions 
%about the reachability time.

%\begin{theorem}[variant of {\cite[Theorem 1]{CFNH16:prob-termination}}]
%\label{thm:old-ranking}
%Let $P$ be an \APP{}, $\sigma$ a scheduler, and 
%$(\Omega,\natfilt,\probm^\sigma)$ the corresponding probability space. Further, 
%let $C$ be a set of 
%configurations of $\pCFG_P$ such 
%that there exist an $\eps>0$ and an $\eps$-ranking supermartingale  
%$\{X_i\}_{i=0}^{\infty}$
%for $\treach{C}$. Then 
%\begin{compactenum}
%\item
%$\probm^{\sigma}(\treach{C}<\infty)=1$, i.e. the set $C$ is reached 
%almost-surely
%\item
%$\E^{\sigma}[\treach{C}]<\E^{\sigma}[X_0]/\eps$.
%\item
%If $\{X_i\}_{i=0}^{\infty}$  has $c$-bounded differences for some $c$, then 
%there exists a concentration bound $B$ for $\treach{C}$ such that $B\leq 
%({E^{\sigma}[X_0]}/{\eps})+1$.
%\end{compactenum}
%\end{theorem}

%\begin{remark}
%We again somewhat generalize the results of~\cite{CFNH16:prob-termination}, 
%where the 
%supermartingale 
%$\{X_i\}_{i=0}^{\infty}$ was additionally required to satisfy the 
%following: there exists $K>0$ such that with probability $1$ it holds $X_i > 
%-K$ for all $i$. We avoid this requirement by use of \emph{stopped ranking 
%supermartingale property} similar to~\cite[Section 10.9]{Williams:book}.
%\end{remark}





%Note that the requirement~\eqref{eq:ranking-sup} can be equivalently expressed 
%by stipulating that $\E[X_{i+1}|\mathcal{F}_i] \leq X_i - \eps\cdot 
%\vec{1}_{\{X_i \geq 0\}}$





