\section{Algorithmic Aspects}

In this section we describe a polynomial-time algorithm for synthesizing linear $\eps$-LexRSM maps in affine probabilistic programs supported by a given linear invariant map $\inv$. The algorithm, based on iterative solving of linear constraints, is an adaptation of a similar algorithm for finding lexicographic ranking functions in non-probabilistic programs~\cite{xxx}. Hence, we provide only a high-level description, focusing on new aspects of our adaptation. 

The main idea is to iteratively synthesize 1-dimensional linear expression map that $1$-rank a subset of generalized transitions. These maps form the individual components of the sought-after $1$-LinLexRSM map. In each iteration, we start with a set $U$ of the yet-unranked generalized transitions. We seek a 1-dimensional LEM which ranks the maximal number of elements if $U$, and is unaffected by the remaining elements of $U$ (here by ranking a generalized transition $\tilde\tau$ we mean ranking it from each configuration $(\loc,\vec{x})$, where $\loc$ is the source location of $\tilde\tau$ and $\vec{x}\in\inv(\loc)$). If no 1-dimensional LEM that would rank at least one element in $U$ exists, then there is no LinLexRSM map for the program. Otherwise, we remove the newly ranked elements from $U$ and continue into the next iteration, until $U$ becomes empty. The process is summarized in Algorithm~\ref{xxx}.

Hence, the main computational task of the algorithm is to check, for a given set of generalized transition $U$, whether there exists a 1-dimensional LEM $\lem$ such that:
\begin{compactenum}
\item for each $\tilde{\tau}\in U$ and  each configuration $(\loc,\vec{x})$ where $\loc$ is the source of $\tilde{\tau}$ and $\vec{x}\in \inv(\loc)$ we have that $\lem(\loc,\vec{x})\geq 0$ and that $\tilde{\tau}$ is unaffected by $\lem$; and
\item there is $\tilde{\tau}\in U$ that is $1$-ranked by $\lem$, from each configuration $(\loc,\vec{x})$ where $\loc$ is the source of $\tilde{\tau}$ and $\vec{x}\in \inv(\loc)\cap\{\vec{x}'\mid \vec{x}'\models \guards(\tilde{\tau})\}$; we then say that $\lem$ ranks $\tilde{\tau}$ w.r.t. $\inv$.
\end{compactenum}
  Moreover, if such an LEM $\lem$ exists, the algorithm has to find one that maximizes the number of gen. transitions in $U$ ranked by it. Both these tasks can be accomplished by the standard  method of linear constraints based on the use of Farkas's lemma, which was widely use for synthesis of termination proofs in both probabilistic and non-probabilistic programs~\cite{xxx}. That is, the algorithm first constructs, for each location $\loc$ a \emph{template} for $\lem$, i.e. an expression of the form $a_1^{\loc}x_1 + \cdots + a_{|\pvars|}^{\loc}x_{|\pvars|} = b^{\loc} $, where $x_1,\dots,x_{|\pvars|}$ are program variables and $a_1^{\loc},\dots,a_{|\pvars|}^{\loc},b^{\loc} $ are yet unknown coefficients. That is, supplying concrete values for all the unknown coefficients yields an LEM. Using the construction provided e.g. in~\cite{xxx} we construct in polynomial time, for each generalized transition $\tilde{\tau}$, a system of linear constraints over set of variables $\{ a_1^{\loc},\dots,a_{|\pvars|}^{\loc},b^{\loc}\mid \loc\in\locs \}\cup \{\eps_{\tilde\tau}\}$