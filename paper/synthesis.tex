\section{Algorithmic Aspects}
\label{sec:algo}

In this section we describe a polynomial-time algorithm for synthesizing linear 
$\eps$-LexRSM maps in affine probabilistic programs supported by a given linear 
invariant map $\inv$. The algorithm, based on iterative solving of linear 
constraints, is a generalization of an algorithm for finding 
lexicographic ranking functions in non-probabilistic 
programs~\cite{ADFG10:lexicographic}. Hence, we provide only a high-level 
description, focusing on the new aspects.

The main idea is to iteratively synthesize 1-dimensional linear expression map that $1$-rank a subset of generalized transitions. These maps form the individual components of the sought-after $1$-LinLexRSM map. In each iteration, we start with a set $U$ of the yet-unranked generalized transitions. We seek a 1-dimensional LEM which ranks the maximal number of elements if $U$, and is unaffected by the remaining elements of $U$ (here by ranking a generalized transition $\tilde\tau$ we mean ranking it from each configuration $(\loc,\vec{x})$, where $\loc$ is the source location of $\tilde\tau$ and $\vec{x}\in\inv(\loc)$). If no 1-dimensional LEM that would rank at least one element in $U$ exists, then there is no LinLexRSM map for the program. Otherwise, we remove the newly ranked elements from $U$ and continue into the next iteration, until $U$ becomes empty. The process is summarized in Algorithm~\ref{algo:linlexrsm}.

Hence, the main computational task of the algorithm is to check, for a given set of generalized transition $U$, whether there exists a 1-dimensional LEM $\lem$ such that:
\begin{compactenum}
\item for each location $\loc\in\locs$ and all $\vec{x}\in \inv(\loc)$ it holds $\lem(\loc,\vec{x})\geq 0$;
\item for each $\tilde{\tau}\in U$ and  each configuration $(\loc,\vec{x})$ where $\loc$ is the source of $\tilde{\tau}$ and $\vec{x}\in \inv(\loc)\cap\{\vec{x}'\mid \vec{x}'\models \guards(\tilde{\tau})\}$ we have that $\tilde{\tau}$ is unaffected by $\lem$ from $(\loc,\vec{x})$; and
\item there is $\tilde{\tau}\in U$ that is $1$-ranked by $\lem$, from each configuration $(\loc,\vec{x})$ where $\loc$ is the source of $\tilde{\tau}$ and $\vec{x}\in \inv(\loc)\cap\{\vec{x}'\mid \vec{x}'\models \guards(\tilde{\tau})\}$; we then say that $\lem$ ranks $\tilde{\tau}$ w.r.t. $\inv$.
\end{compactenum}
  Moreover, if such an LEM $\lem$ exists, the algorithm has to find one that 
  maximizes the number of gen. transitions in $U$ ranked by it. Both these 
  tasks can be accomplished by the standard  method of linear constraints based 
  on the use of Farkas's lemma, which was widely use for synthesis of 
  termination proofs in both probabilistic and non-probabilistic 
  programs~\cite{DBLP:conf/tacas/ColonS01,DBLP:conf/vmcai/PodelskiR04,SriramCAV,CFNH16:prob-termination}.
   That is, the algorithm first constructs, for each location $\loc$ a 
  \emph{template} for $\lem$, i.e. an expression of the form $a_1^{\loc}x_1 + 
  \cdots + a_{|\pvars|}^{\loc}x_{|\pvars|} = b^{\loc} $, where 
  $x_1,\dots,x_{|\pvars|}$ are program variables and 
  $a_1^{\loc},\dots,a_{|\pvars|}^{\loc},b^{\loc} $ are yet unknown 
  coefficients. That is, supplying concrete values for all the unknown 
  coefficients yields an LEM. Now the conditions (1) and (2) above can be 
  expressed using linear constraints on the coefficients. More precisely, using 
  the construction provided e.g. in~\cite{SriramCAV,CFNH16:prob-termination} 
  (which includes a use of the Farkas's lemma) we construct in polynomial time, 
  for each generalized transition $\tilde{\tau}$, a system of linear 
  constraints $\linsystem_{\tilde{\tau}}$ over set of variables $\{ 
  a_1^{\loc},\dots,a_{|\pvars|}^{\loc},b^{\loc}\mid \loc\in\locs \}\cup 
  \{\eps_{\tilde\tau}\}\cup F$, where $F$ is the set of fresh variables (not 
  appearing in any template) and $\eps_{\tilde\tau} $ is constrained to be 
  non-negative. Each solution of the system $\linsystem_{\tilde{\tau}}$ yields 
  a LEM which satisfies the constraints (1) and (2) for $\tilde{\tau}$. 
  Moreover, each solution of $\linsystem_{\tilde{\tau}}$ yields a LEM which 
  $\eps_{\tilde{\tau}}$-ranks $\tilde{\tau}$. To find a LEM which satisfies all 
  constraints (1)--(3) as well as maximizes the number of 1-ranked elements of 
  $U$ it is sufficient to construct $\linsystem_{\tilde{\tau}}$ for each 
  $\tilde{\tau}\in U$ and solve the following linear program $\lp_{U}$:
  \begin{align*}
\text{maximize }  &\sum_{\tilde{\tau} \in U} \eps_{\tilde\tau} \text{ subject to constraints}\\[0.3cm]
&\linsystem_{\tilde{\tau}}\,;\quad\quad \quad\quad\tilde{\tau}\in U\\
&0 \leq \eps_{\tilde{\tau}} \leq 1\,; \quad \tilde{\tau}\in U
  \end{align*}
  
Each system $\linsystem_{\tilde{\tau}}$ is constructed in such a way that if it admits a solution with some $\eps_{\tilde{\tau}}$ positive, then decreasing the value of $\eps_{\tilde{\tau}}$ in that solution to any non-negative value still yield a valid solution (this corresponds to the fact that if some transition is $\eps$-ranked by $\lem$, than it is $\eps'$-ranked by $\lem$ for each $0\leq \eps'\leq \eps$). Morever, each solution where $\eps_{\tilde{\tau}}$ is positive can be rescaled into another solution in which $\eps_{\tilde{\tau}}$ is at least $1$. It follows that of $\lp_{U}$ has at least one feasible solution, then it has an optimal solution in which each $\eps_{\tilde{\tau}}$ is either $0$ or $1$. If the system does not have a solution or all the $\eps_{\tilde{\tau}}$ are equal to zero then there is no LEM satisfying (1)--(3). Otherwise, the optimal solution of $\lp_{U}$ yields a LEM $\lem$ which satisfies (1)--(3) and maximizes the number of 1-ranked elements of $U$.

This polynomial-time linear-programming step is used as a sub-procedure in Algorithm~\ref{algo:linlexrsm} for LinLexRSM synthesis. 

\begin{algorithm}
\SetKwInOut{Input}{input}\SetKwInOut{Output}{output}
\DontPrintSemicolon

\Input{An \APP{} $\program$ together with an invariant map $\inv$.}
\Output{A multi-dimensional LinLexRSM if it exists, otherwise ``No LinLexRSM''}
$U \leftarrow \text{ all generalized transitions of } \pCFG_{\program}$\;
$d\leftarrow 0$\;
\While{$U$ is non-empty}{
	$d \leftarrow d+1$\;
	construct and solve $\lp_{U}$\;
	\If{$\lp_U$ does not have feasible solution or optimal value is $0$}{\Return{No LinLexRSM}}
	\Else{$\sol\leftarrow$ optimal solution of $\lp_U$\;
		$\lem_d \leftarrow $ the LEM $\lem$ induced by $\sol$\;
		$U\leftarrow U \setminus \{\tilde{\tau}\mid \eps_{\tilde{\tau}}=1 \text{ in }\sol \}$}
}
\Return{$(\lem_1,\dots,\lem_d)$}
\caption{Synthesis of LinLexRSMs for \APP{}s}
\label{algo:linlexrsm}
\end{algorithm}

Both soundness and relative completeness of Algorithm~\ref{algo:linlexrsm} 
can be proved by generalization of arguments presented 
in~\cite{ADFG10:lexicographic}. 
\begin{theorem}
\label{thm:algo}
\label{THM:ALGO}
	Suppose that Algorithm~\ref{algo:linlexrsm} is run on an \APP{} $\program$ 
	together with linear invariant map $\inv$. If the algorithm returns a 
	$d$-dimensional LEM $\vec{\lem}=(\lem_1,\dots,\lem_d)$, then $\vec{\lem}$ 
	is a $1$-LinLexRSM map for $\program$ supported by $\inv$. Conversely, if 
	the algorithm returns ``No LinLexRSM'', then for any $d'\in \Nset$ and 
	$\eps>0$ there is no $d'$-dimensional $\eps$-LinLexRSM for $\program$  
	supported by $\inv$. If guards of all conditionals and loops in $\APP{}$ 
	are linear assertion, then the
	algorithm runs in time polynomial in size of 
	$\program$ and $\inv$.
\end{theorem}
\begin{proof}[Proof (Key Ideas)]
For soundness, let $U_i$ and $U_{i}$ denote the content of $U$ just before the $i$-th iteration of the while loop. Then each $\tilde{\tau}\in U_{i}\setminus U_{i+1}$ is $1$-ranked by $\lem_i$, for each $1\leq i \leq d$. Since $U_{d}=\emptyset$, each generalized transition of $\pCFG_{\program}$ is 1-ranked by some component of $\vec{\lem}$. Non-negativity of each $\lem_i$ is ensured directly by $\lp_{U_i}$. Hence, it remains to show that each $\tilde{\tau}\in U_i$ is unaffected by $\lem_1,\dots,\lem_{i-1}$. But this follows from $U_i \subseteq U_{i-1}\subseteq \cdots\subseteq U_1$ and from the fact that each gen. transition in $U_j$ is unaffected by $\lem_j$.

Proving completeness is more intricate; as pointed out 
in~\cite{ADFG10:lexicographic}, one needs to show that the greedy strategy of 
selecting LEM that ranks maximal number of remaining transitions does not cut 
off some possible LinLexRSMs. In~\cite{ADFG10:lexicographic} the completeness 
of the greedy strategy is proved by using several geometric arguments that 
exploit the fact that the underlying programs are affine. The same geometric 
properties hold for our generalization to \APP{}s (all ranking conditions in 
Definitions~\ref{def:rank1} and~\ref{def:rank2} are linear in program 
variables), so the result is easily transferable. The complexity argument is 
rather standard and we present it in~\AppendixMaterial.
\end{proof}

%\