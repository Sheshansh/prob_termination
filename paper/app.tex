\section{Details of Program Syntax}
In this subsection we present the details of the syntax of (affine) 
probabilistic 
programs.

Recall that $\vars$ %and $\mathcal{R}$ 
is a collection of 
\emph{variables}.
%and \emph{random} variables, respectively. 
Moreover, let $\mathcal{D}$ be a set of \emph{probability distributions} on 
real numbers.
The abstract syntax of affine
probabilistic programs (\APP s)
is given by the grammar in Figure~\ref{fig:syntax}, where
the expressions $\langle \mathit{pvar}\rangle$ and $\langle
\mathit{dist}\rangle$  range over $\vars$ and $\mathcal{D}$, respectively.
We allow for non-deterministic assignments, expressed by a statement $x:=
\text{\textbf{ndet($\mathit{dom}$)}}$, where $\mathit{dom}$ is a \emph{domain
specifier} determining the set from which the value can be chosen: for general 
programs it can be any Borel-measureable set, for \APP{}s it has to be an 
interval (possibly of infinite length).  
The grammar is such that $\langle \mathit{expr} \rangle$ %and $\langle
%\mathit{rexpr} \rangle$ 
may evaluate to an arbitrary affine expression over the
program variables.
%, and the program and random variables, respectively (note that
%random variables can only be used in the RHS of an assignment). 
Next, $\langle
\mathit{bexpr}\rangle$ may evaluate to an arbitrary propositionally linear
predicate. 

For general (not necessary affine) \PP{}s we set $\langle\mathit{expr}\rangle$ 
to be the set of all expressions permitted by the set of mathematical 
operations of the underlying language. Similarly, $\langle\mathit{bexpr} 
\rangle$ is the set of all predicates, as defined in Section~\ref{sec:prelim}.

The guard of each if-then-else statement is either $\star$, 
representing a (demonic) non-deterministic choice between the branches,
a keyword \textbf{prob}($p$), where $p\in [0,1]$ is a number given in decimal
representation (represents a probabilistic choice, where  the if-branch is
executed with probability $p$ and the then-branch with probability $1-p$), or
the guard is a propositionally linear predicate, in which case the statement
represents a standard deterministic conditional branching.

%We assume that each \APP{} $\program$ is preceded by an initialization 
%preamble 
%in 
%which 
%each variable appearing in $\program$ is assigned some concrete number. 
Regarding distributions, for each $d\in \mathcal{D}$ we assume the
existence of a program primitive denoted by '\textbf{sample($d$)}' implementing 
sampling from $d$. In practice, the distributions appearing in a program would 
be those for which sampling is 
provided by suitable libraries (such as 
uniform distribution over some interval, Bernoulli, geometric, etc.), 
but we 
abstract away from these implementation details. For the purpose of our 
analysis, it is sufficient that for each distribution $d$ appearing in the 
program the following characteristics: expected value $\expv[d]$ of $d$ and a 
set $SP_d$ containing \emph{support} of $d$  (the support of $d$ is the 
smallest 
closed set of real numbers whose complement has probability zero 
under $d$). \footnote{In 
particular, 
a support of a \emph{discrete} probability 
distribution $d$ is simply the at most countable set of all points on a real 
line that have positive probability under $d$. For continuous distributions, 
e.g. a normal distribution, uniform, etc., the support is typically either 
$\Rset$ or some closed real interval. } For \APP{}s, $SP_d$ is required to be 
an 
interval. 
\begin{figure}
\begin{align*}
\langle \mathit{stmt}\rangle &::= 
%\,\langle\mathit{pvar}\rangle
%\,\text{'$:=$'}\, \langle\mathit{rexpr} \rangle \mid 
%\langle\mathit{pvar}\rangle \,\text{'$:=$}\,
%\text{\textbf{ndet($\langle\mathit{dom}\rangle$)}'}  \\
\langle \mathit{assgn} \rangle \mid \text{'\textbf{skip}'} \mid 
\langle\mathit{stmt}\rangle \, \text{';'} \, \langle \mathit{stmt}\rangle \\
%
&\mid   \text{'\textbf{if}'} \,
\langle\mathit{ndbexpr}\rangle\,\text{'\textbf{then}'} \, \langle
\mathit{stmt}\rangle \, \text{'\textbf{else}'} \, \langle \mathit{stmt}\rangle
\,\text{'\textbf{fi}'}
\\
%
&\mid  \text{'\textbf{while}'}\, \langle\mathit{bexpr}\rangle \,
\text{'\textbf{do}'} \, \langle \mathit{stmt}\rangle \, \text{'\textbf{od}'}
\\
\langle \mathit{assgn} \rangle &::= 
\,\langle\mathit{pvar}\rangle
\,\text{'$:=$'}\, \langle\mathit{expr} \rangle \mid 
\langle\mathit{pvar}\rangle \,\text{'$:=$}\,
\text{\textbf{ndet($\langle\mathit{dom}\rangle$)}'}
\\
&\mid \langle\mathit{pvar}\rangle \,\text{'$:=$}\,
\text{\textbf{sample($\langle\mathit{dist}\rangle$)}'}
%
%
%
%
%&
%
%\\
%
%\\
\\
\vspace{0.5\baselineskip}
%\\
%\langle\mathit{var} \rangle &::= \langle\mathit{pvar} \rangle \mid 
%\langle\mathit{rvar}\rangle
%\\
%%\vspace{\baselineskip}
%\vspace{0.5\baselineskip}
%\\
\langle\mathit{expr} \rangle &::= \langle \mathit{constant} \rangle \mid
\langle\mathit{pvar}\rangle
%\\
\mid \langle \mathit{constant} \rangle \,\text{'$\cdot$'} \,
\langle\mathit{pvar}\rangle
\\
%
&\mid \langle\mathit{expr} \rangle\, \text{'$+$'} \,\langle\mathit{expr} \rangle
\mid \langle\mathit{expr} \rangle\, \text{'$-$'} \,\langle\mathit{expr} \rangle
%\\
%%
%\vspace{0.5\baselineskip}
%%\\
%\langle\mathit{rexpr} \rangle &::= \langle\mathit{expr} \rangle \mid 
%\langle\mathit{rvar}\rangle 
%%\mid
%% \langle\mathit{pvar}\rangle 
%\mid\langle\mathit{constant} \rangle \,\text{'$*$'} \,
%\langle\mathit{rvar}\rangle \\
%%
%&\mid
%%\langle\mathit{constant} \rangle \,\text{'$*$'} \, \langle\mathit{pvar}\rangle
%%\mid 
%\langle\mathit{rexpr} \rangle\, \text{'$+$'} \,\langle\mathit{rexpr} \rangle
%\mid \langle\mathit{rexpr} \rangle\, \text{'$-$'} \,\langle\mathit{rexpr}
%\rangle
%%\\
%%\vspace{0.5\baselineskip}
%%\\ \langle\mathit{bexpr} \rangle &::= \langle \mathit{predicate}\rangle \mid 
%%\neg \langle\mathit{predicate}\rangle
\\
%
%\vspace{0.5\baselineskip}
%\langle \mathit{dom} \rangle &::= \text{'\textbf{Int}'} \mid
%\text{'\textbf{Real}'} \mid
%\text{'\textbf{Int}$[\langle\mathit{constant}\rangle,\langle\mathit{constant}\rangle]$'}
% \\ 
%% 
% &\mid
%\text{'\textbf{Real}$[\langle\mathit{constant}\rangle,\langle\mathit{constant}\rangle]$'}
%\mid \langle \mathit{dom} \rangle \text{'\textbf{or}'}\langle \mathit{dom}
%\rangle
%\\
%
%\vspace{0.5\baselineskip}
%\\
\langle \mathit{bexpr}\rangle &::=  \langle \mathit{affexpr} \rangle \mid
\langle \mathit{affexpr} \rangle \, \text{'\textbf{or}'} \,
\langle\mathit{bexpr}\rangle
\vspace{0.5\baselineskip}
\\
%
%\vspace{\baselineskip}
%\\
\langle\mathit{affexpr} \rangle &::=  \langle\mathit{literal} \rangle\mid
\langle\mathit{literal} \rangle\, \text{'\textbf{and}'}
\,\langle\mathit{affexpr} \rangle
\\
%
\langle\mathit{literal} \rangle &::= \langle\mathit{expr} \rangle\,
\text{'$\leq$'} \,\langle\mathit{expr} \rangle \mid \langle\mathit{expr}
\rangle\, \text{'$\geq$'} \,\langle\mathit{expr} \rangle
\\
%
&\mid \neg \langle \mathit{literal} \rangle
\\
%
%\vspace{0.5\baselineskip}
%\\
\langle\mathit{ndbexpr} \rangle &::= {\star}\mid
\text{'\textbf{prob($p$)}'} \mid \langle\mathit{bexpr} \rangle
\end{align*}
\caption{Syntax of affine probabilistic programs (\APP 's).}
\label{fig:syntax}
\end{figure}

%The syntax is such that $\langle \mathit{expr} \rangle$ and $\langle
%\mathit{rexpr} \rangle$ may evaluate to an arbitrary affine expression over the
%program variables, and the program and random variables, respectively (note
%that random variables can only be used in the RHS of an assignment). Next,
%$\langle \mathit{bexpr}\rangle$ may evaluate to an arbitrary disjunction of
%polyhedral constraints (i.e. conjunctions of linear inequalities and equalities
%over the program variables).



%Additionally to the program body generated by the grammar, we assume that each
%program has a preamble in which both program and random variables are declared.
%We assume that every program variable is initialized to some fixed constant
%upon declaration, and that for each of the random variables its
%For random variables we need to suitably specify their distribution in the
%preamble. This amounts to either directly specifying the joint distribution of
%these variables, or stipulating the variables to be stochastically independent
%and specifying a probability distribution for each one of them. We point out
%that the only parameters of the program's random variables that we use in the
%process of deriving ranking supermartingales are their expected values (which
%we assume to be well-defined and finite) and, for deriving concentration
%inequalities, also bounds on their range. \PN{Is this true? Do we not need
%bounds on their range for as termination as well?} The exact distributions are
%only needed for the actual execution of the program.


%\begin{example}\label{ex:prog}
%We present an example of an affine probabilistic program shown in
%Figure~\ref{ex:prob}.
%The program variable is $x$, and there is a while loop, where given a
%probabilistic
%choice one of two statement blocks $Q_1$ or $Q_2$ is executed.
%The block $Q_1$ (resp. $Q_2$) is executed if the probabilistic choice is at
%least
%$0.6$ (resp. less than $0.4$).
%The statement block $Q_1$ (resp., $Q_2$) is an angelic (resp. demonic)
%conditional
%statement to either increment or decrement $x$.
%\lstset{language=affprob}
%\lstset{tabsize=3}
%\newsavebox{\affproblist}
%\begin{lrbox}{\affproblist}
%\begin{lstlisting}[mathescape]
%$x:=0$;
%while $x \geq 0$ do
%	if prob(0.6) then
%		if angel then
%			$x:=x+1$
%		else
%			$x:=x-1$
%		fi
%	else
%		if demon then
%			$x:=x+1$
%		else
%			$x:=x-1$
%		fi
%	fi
%od
%\end{lstlisting}
%\end{lrbox}
%\begin{figure}[t]
%%%\centering
%\usebox{\affproblist}
%\caption{An example of a probabilistic program}\label{ex:prob}
%\end{figure}
%\end{example}


\section{Details of Program Semantics}


\begin{remark}[Use of random variables]
%\label{rem:randuse}
In the paper we sometimes work with random variables 
that are functions of the type $R\colon\Omega \rightarrow S$ for some finite 
set $S$. These can be captured by the definition given in 
Section~\ref{sec:prelim} by identifying the 
elements of $S$ with distinct real numbers.\footnote{This is equivalent to 
saying that a function $R\colon \Omega\rightarrow S$, with $S$ finite, is a 
random variable if for each $s\in S$ the set $\{\omega\in \Omega\mid 
R(\omega)=s\}$ belongs to $\mathcal{F}$.} The exact choice of numbers is 
irrelevant in such a case, as we are not interested in, e.g. computing expected 
values of such random variables, or similar operations. 
\end{remark}


\paragraph*{From Programs to pCFGs}
To every probabilistic program $P$ we can assign a pCFG $\pCFG_P$ whose 
locations correspond to the values of the
program counter of $P$ and whose transition relation captures the behaviour of
$P$. We illustrate the construction for \APP{}s, for general programs it is 
similar. To obtain $\pCFG_{P}$, we first rename 
the variables in $P$ to 
$x_1,\dots,x_n$, where $n$ is the number of distinct variables in the program. 
The
construction of $\pCFG_P$ can be described inductively.
For each program $P$ the pCFG $\pCFG_P$ contains two distinguished
locations, $\ell^{\lin}_{P}$ and $\ell^{\lout}_{P}$, the latter one being always
deterministic, that intuitively represent the state of the program counter
before and after executing $P$, respectively. In the following, we denote by 
$\id_1$ a function such that for each $\vec{x}$ we have 
$\id_{1}(\vec{x})=\vec{x}[1]$.
\begin{compactenum}
\item {\em Deterministic Assignments and Skips.}
For $P= {x_j}{:=}{E}$ where $x_j$ is a program variable and $E$ is an 
expression, or $P = \textbf{skip}$, the pCFG $\pCFG_P$ consists only of
%these two (deterministic) locations
locations $\ell^{\lin}_P$ and $\ell^{\lout}_P$ (first assignment location, 
second one deterministic) and a
transition $(\ell^{\lin}_{P},\ell^{\lout}_P)$. In the first case, 
$U(\ell^{\lin}_{P},\ell^{\lout}_P)=(j,E)$.
\item {\em Probabilistic and Non-Deterministic Assignemnts}
For $P= {x_j}{:=}{\textbf{sample($d$)}}$ where $x_j$ is a program variable and 
$d$ is a distribution, the pCFG $\pCFG_P$ consists locations $\ell^{\lin}_P$ 
and $\ell^{\lout}_P$ and a
transition $\tau=(\ell^{\lin}_{P},\ell^{\lout}_P)$ with $U(\tau)=(j,d)$. For 
$P= 
{x_j}{:=}{\textbf{ndet($\mathit{dom}$)}}$, the construction is similar, with 
the only transition being $\tau=(\ell^{\lin}_{P}\ell^{\lout}_P)$ and 
$U(\tau)=(j,D)$, where 
$D$ is 
the set specified by the domain specifier $\mathit{dom}$.

\item {\em Sequential Statements.}
For $P = Q_1;Q_2$ we take the pCFGs $\pCFG_{Q_1}$, $\pCFG_{Q_2}$ and
join them by identifying the location $\ell^{\lout}_{Q_1}$ with
$\ell^{\lin}_{Q_2}$, putting $\ell^{\lin}_{P}=\ell^{\lin}_{Q_1}$ and
$\ell^{\lout}_{P}=\ell^{\lout}_{Q_2}$.

\item {\em While Statements.}
For $P = \textbf{while $\phi$ do }Q \textbf{ od}$ we add a new deterministic
location $\ell^{\lin}_{P}$ which we identify with $\ell^{\lout}_{Q}$, a new
deterministic location $\ell^{\lout}_{P}$, and transitions
$\tau=(\ell^{\lin}_{P},\ell^{\lin}_{Q})$,
$\tau'=(\ell^{\lin}_{P},\ell^{\lout}_{P})$ such that $G(\tau)=\phi$ and
$G(\tau')=\neg\phi$.

\item {\em If Statements.}
Finally, for $P = \textbf{if $\mathit{ndb}$ then }Q_1 \textbf{ else } Q_2
\textbf{ fi}$ we add a new location $\ell^{\lin}_{P}$ (which is not an 
assignment location) together with two
transitions $\tau_1 = (\ell^{\lin}_{P},\ell^{\lin}_{Q_1})$, $\tau_2 =
(\ell^{\lin}_{P},\ell^{\lin}_{Q_2})$, and we identify the locations 
$\ell^{\lout}_{Q_1}$ and $\ell^{\lout}_{Q_1}$ with $\ell^{\lout}_{P}$. (If both
$Q_j$'s consist of a single statement, we also identify $\ell^\lin_{P}$ with 
$\ell^{\lin}_{Q_j}$'s.) In this
case the newly added location $\ell^\lin_{P}$ is non-deterministic branching if 
and only 
if
$ndb$ is the keyword '$\star$'. If
$\mathit{ndb}$ is of the form $\textbf{prob($p$)}$, the location $\ell^\lin_{P}$
is probabilistic branching with $\probdist_{\ell^\lin_{P}}(\tau_1)=p$ and
$\probdist_{\ell^\lin_{P}}(\tau_2)=1-p$. Otherwise (i.e. if $\mathit{ndb}$ is a
predicate), $\ell^\lin_{P}$ is a deterministic location
with $G(\tau_1)=\mathit{ndb}$ and $G(\tau_2)=\neg \mathit{ndb}$.
\end{compactenum}
Once the pCFG $\pCFG_P$ is constructed using the above rules, we put
$G(\tau)=\textit{true}$ for all transitions $\tau$ outgoing from deterministic
locations whose guard was not set in the process, and finally we add a self-loop
on the location $\ell^{\lout}_P$. This ensures that the assumptions in
Definition~\ref{def:stochgame} are satisfied.
Furthermore note that for pCFG obtained for a program $P$, since the only
branching is if-then-else branching, every location $\loc$ has at most two
successors $\loc_1,\loc_2$.



