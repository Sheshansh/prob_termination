
\vspace{-4em}
\section{Related Work}
\vspace{-0.5em}
%%In this section we discuss the related works.

%%\smallskip
\noindent{\em Probabilistic programs and termination.}
In early works the termination for concurrent probabilistic programs was studied 
as fairness~\cite{SPH84}, which ignored precise probabilities.
For countable state space a sound and complete characterization of almost-sure termination 
was presented in~\cite{HS85}, but nondeterminism was absent.
A sound and complete method for proving termination of finite-state programs
was given in~\cite{EGK12}.
For probablistic programs with countable state space and without 
nondeterminism, the {\em Lyapunov ranking functions} provide a sound and 
complete method to prove positive termination~\cite{BG05,Foster53}.
For probabilistic programs with nondeterminism, but restricted to discrete probabilistic
choices, the termination problem was studied 
in~\cite{MM04,MM05}.
The RSM-based (ranking supermatingale-based) approach extending ranking functions was first 
presented in~\cite{SriramCAV} for probabilistic programs without non-determinism,
but with real-valued variables, and its extension for probabilistic programs
with non-determinism has been studied 
in~\cite{HolgerPOPL,CF17,CFNH16:prob-termination,CFG16,CNZ17,MM16:proofrule-arxiv}. 
Supermartingales were also 
considered for other liveness and safety 
properties~\cite{CVS16:martingale-recurrence-persistence,BEFH16:doob}.
While all these results deeply clarify the role of RSMs for probabilistic programs, 
the notion of lexicographic RSMs to obtain a practical approach for termination 
analysis for probabilistic programs has not been studied before, which we consider in 
this work.

\noindent{\em Compositional a.s. termination proving.} 
A 
compositional rule for proving almost-sure termination was studied 
in~\cite{HolgerPOPL} under the uniform integrability assumption. 
In~\cite{MM05}, a soundness of the probabilistic variant rule is proved for 
programs with finitely many configurations.


\noindent{\em Other approaches.}
Besides RSMs, other approaches has also been considered for probabilistic programs.
Logical calculi for reasoning about properties of 
probabilistic programs (including termination) were studied 
in~\cite{Kozen:prob-semantics,FH:prdl,Kozen:probabilistic-PDL,Feldman:propositional-probdl}
 and extended to programs with demonic non-determinism 
 in~\cite{MM04,MM05,KKMO16:wp-expected-runtime,OKKM16:recursive-prob-wp-calculus,GKI14:prob-semantics,
  DBLP:conf/sas/KatoenMMM10}. However, none of these approaches is readily 
  automatizable.
A sound approach~\cite{DBLP:conf/sas/Monniaux01} for almost-sure termination 
is to explore the exponential decrease of probabilities upon 
bounded-termination 
through abstract interpretation~\cite{DBLP:conf/popl/CousotC77}. A method for 
a.s. termination of weakly finite programs (where number of reachable 
configurations is finite from each initial configuration) based on 
\emph{patterns} was presented in~\cite{EGK12}.
 

\noindent{\em Non-probabilistic programs.}
Termination analysis of non-probabilistic programs has also been extensively 
studied~\cite{PR04:transition-invariants, 
CPR06:terminator,DBLP:conf/cav/BradleyMS05,DBLP:conf/tacas/ColonS01, 
DBLP:conf/vmcai/PodelskiR04,DBLP:conf/pods/SohnG91,BMS05b,LJB01,KSTW10:compositional-transition-invariants,
 CPR11:termination-cacm}.
Ranking functions are at the heart of the termination analysis, and lexicographic 
ranking function has emerged as one of the most efficient and practical approaches
for termination analysis~\cite{CSZ13,ADFG10:lexicographic,GMR15:rank-extremal}, 
being used e.g. in 
the prominent \texttt{T2} temporal prover~\cite{BCIKP16:T2}.
In this work we extend lexicographic ranking functions to probabilistic 
programs,
and present lexicographic RSMs for almost-sure termination analysis of probabilistic programs
with non-determinism. Theoretical complexity of synthesizing lexicographic 
ranking functions in non-probabilistic programs was studied 
in~\cite{BG13:integer-ranking,BG15:lexicographic-complexity}.


\vspace{-0.5em}
\section{Conclusion and Future Work}
\vspace{-0.5em}
In this work we considered lexicographic RSMs for termination analysis of probabilistic
programs with non-determinism.
We showed it presents a sound approach for almost-sure termination, that is
algorithmically efficient, enables compositional reasoning about termination, 
and leads to approach that can handle realistic programs.
There are several interesting directions of future work.
Lexicographic ranking functions has been considered in several works to 
provide different practical methods for analysis of non-probabilistic programs.
First, while our work presents the foundations of lexicographic RSMs for probabilistic 
programs, extending other practical methods based on lexicographic ranking 
functions to 
lexicographic RSMs is an interesting direction of future work.
Second, while our algorithmic approaches focus on the linear case, it would be interesting
to consider other non-linear, and polynomial functions in the future.





