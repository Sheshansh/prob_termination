\section{Applying Lexicographic Supermartingales to Probabilistic Programs}
\label{sec:lex-programs}

We now discuss how to leverage the mathematical results of the previous section 
to provide a sound proof rule for almost-sure termination of probabilistic 
programs. Hence, for the rest of this section we fix a \PP{} $\program$ and the 
associated pCFG 
$\pCFG_\program=(\locs,\pvars,\locinit,\vecinitset,\transitions,\updates,\probdist,\guards)$.

We aim to define a function assigning a non-negative vector to each 
configuration (so called measurable map) such that in each point of 
computation, the expected value of the function after performing one more 
computational step is smaller (in lexicographic ordering) than the current one. 
We formalize this property below.
%This property can be formalized using the standard  notion of 
%\emph{pre-expectation}~\cite{xxx}.

\begin{definition}[Measurable Maps and Linear Expression Maps]
A 1-dimensional \emph{measurable map} for a \PP{} $\program$ is a  
real-valued function $\lem$ 
assigning to each program location $\loc$ of $\pCFG_P$ a Borel-measurable function $\lem(\loc)$  of program variables, i.e. each $\lem(\loc)$  is a function of type $\Rset^{|\pvars|}\rightarrow \Rset$. As a special case, if all the functions $\lem(\loc)$ are affine, then we call $\lem$ a 1-dimensional \emph{linear expression map (LEM)}. 
Am $n$-dimensional measurable/linear expression map is a vector $\vec{\lem}=(\lem_1,\dots,\lem_n)$ of 1-dimensional measurable/linear expression maps. 
\end{definition}

Each 1-dimensional measurable map $\lem$ and location $\loc$ determines a function $\lem(\loc)$ 
which takes as an argument an $|\pvars|$-dimensional vector. We use $\lem(\loc,\vec{x})$ as a shorthand 
notation for $\lem(\loc)(\vec{x})$.

We now formalize the notion of a transition in a pCFG being ranked by a 
measurable map. We first define this notion for transitions that do not go out 
of a probabilistic branching location, as these require a special treatment.

\begin{definition}
\label{def:rank1}
Let $\lem$ be a measurable map, $(\loc,\vec{x})$ be a configuration such that 
$\loc\not\in \locsPB$ and let 
$\tau=(\loc,\loc')$ be a 
transition outgoing from $\loc$. For an $\eps\geq 0$ we say that $\tau$ is 
\emph{$\eps$-ranked by $\lem$ from $(\loc,\vec{x})$} if the following 
conditions are 
satisfied, depending on the type of $\loc$:
\begin{compactitem} %\itemsep1pt \parskip0pt \parsep0pt
%		\item 
%		if $\loc$ is a probabilistic branching location, then
%		$$\preexp{\lem}(\loc,\vec{x}):=\sum_{(\loc,\loc')\in\transitions} 
%		Pr_{\loc}\left(\loc,\loc'\right)\cdot
%		\lem(\loc',\vec{x});$$
		\item 
		if $\loc$ is a deterministic or non-deterministic branching location, 
		then
		$$\lem(\loc',\vec{x}) \leq 
		\lem(\loc,\vec{x})-\eps;$$
		
		%\item 
		
%%$\mathrm{pre}_\eta(\loc,\mathbf{x}):=\min_{(\loc,id,\ell')\in\transitions}\eta\left(\loc',\mathbf{x}\right)$
		% if $\loc$ is an angelic location;
%		\item 
%		if $\loc$ is a deterministic location, then there is 
%exactly one 
%		transition
%		$\tau=(\loc,\loc')$ such that $\vec{x}\models G(\tau)$. We put 
%$$\preexp{\lem}(\loc,\vec{x}):= \lem(\loc',\vec{x})$$.
		\item 
		If $\loc$ is an assignment location, then we distinguish 
		three cases, depending on $\updates(\tau)=(j,\up)$ (recall that $\up$ 
is an update element):
		\begin{compactitem}
			\item If $\up\colon \Rset^{|\pvars|}\rightarrow \Rset$ is a 
Borel-measurable function, then we require
			$$\lem(\loc',\vec{x}(j\leftarrow 
\up(\vec{x}))) \leq \lem(\loc,\vec{x})-\eps$$
			\item If $\up$ is a distribution $d$, then we require $$ 
			\lem(\loc',\vec{x}(j\leftarrow \E[d])) \leq 
			\lem(\loc,\vec{x})-\eps,$$ 
			where $\E[d]$ is the 
expected value of the 
			distribution $d$.
			\item 
			If $\up$ is a set, then we require $$\sup_{a\in\up}
			\lem(\loc',\vec{x}(j \leftarrow a)) \leq \lem(\loc,\vec{x})-\eps.$$
		\end{compactitem}
		
%%$\mathrm{pre}_\eta(\loc,\mathbf{x}):=\eta(\loc',\expv_{\rvars}\left(f(\vec{x},\mathbf{r})\right))$
		% if $\loc$ is a deterministic location, 
%$(\loc,f,\loc')\in\transitions$ and 
		%$\mathbf{x}\in G(\loc,f,\loc')$, where 
		%$\expv_{\rvars}\left(f(\vec{x},\mathbf{r})\right)$ is the expected 
%value of 
		%$f(\vec{x},\cdot)$. %on $\rvars$ if $\vec{x}$ is treated as a constant 
%%vector.
	\end{compactitem}
\end{definition}

Since ranking supermartingales are required to decrease on average, for 
individual transitions outgoing from $\locsPB$ it does not make sense to say 
that they are ranked or not. Instead, for each $\loc\in\locsPB$ we consider all 
outgoing transitions together.

\begin{definition}
\label{def:rank2}
Let $\lem$ be a measurable map, and let $(\loc,\vec{x})$ be a configuration with
$\loc\in \locsPB$. For an $\eps\geq 0$ we say that $\loc$ is $\eps$-ranked by 
$\lem$ from $(\loc,\vec{x})$ if 
$$\sum_{(\loc,\loc')\in\transitions} 
		Pr_{\loc}\left(\loc,\loc'\right)\cdot
		\lem(\loc',\vec{x}\leq \lem(\loc,\vec{x})-\eps.$$
\end{definition}

To capture the specific of $\locsPB$, we introduce the 
notion of \emph{generalized transition}.

\begin{definition}
A generalized transition of a pCFG $\pCFG$ is either a transition of $\pCFG$ 
outgoing 
from location not in $\locsPB$ or a location $\loc\in\locsPB$.
\end{definition}

Intuitively, we represent the set of transitions outgoing from $\loc\in\locsPB$ 
by the source location $\loc$.  For generalized transitions 
$\tilde{\tau}=\loc\in\locsPB$ we say that $\tilde{\tau}$ is outgoing from 
$\loc$.

Definitions~\ref{def:rank1} and~\ref{def:rank2} 
define when is a generalized transition $\eps$-ranked by $\lem$ from 
configuration $(\loc,\vec{x})$. We say that a 
generalized transition is \emph{unaffected} by $\lem$ from $(\loc,\vec{x})$ if 
it is $0$-ranked by $\lem$ from $(\loc,\vec{x})$.

%\begin{definition}[Pre-Expectation]
%	Let $\lem$ 
%	be 1-dimensional measurable map for a \PP{} $\program$.
%	%Let $\locs$ 
%	%be the set of locations of $\pCFG_P$ and $\pvars$ its set of program 
%	%variables. 
%	The 
%	pre-expectation of $\lem$ is a function $\preexp{\lem}\colon \locs\times 
%	\Rset^{|\pvars|} \rightarrow \Rset\cup\{\infty\}$ defined as follows:
%	\begin{compactitem} %\itemsep1pt \parskip0pt \parsep0pt
%		\item 
%		if $\loc$ is a probabilistic branching location, then
%		$$\preexp{\lem}(\loc,\vec{x}):=\sum_{(\loc,\loc')\in\transitions} 
%		Pr_{\loc}\left(\loc,\loc'\right)\cdot
%		\lem(\loc',\vec{x});$$
%		\item 
%		if $\loc$ is a non-deterministic branching location, then
%		$$\preexp{\lem}(\loc,\vec{x}):=
%		\max_{(\loc,\loc')\in\transitions}\lem(\loc',\vec{x});$$
%		
%		%\item 
%		
%%%$\mathrm{pre}_\eta(\loc,\mathbf{x}):=\min_{(\loc,id,\ell')\in\transitions}\eta\left(\loc',\mathbf{x}\right)$
%		% if $\loc$ is an angelic location;
%		\item 
%		if $\loc$ is a deterministic location, then for each $\vec{x}$ the 
%value 
%		$\preexp{\lem}(\loc,\vec{x})$ is determined as follows: there is 
%exactly one 
%		transition
%		$\tau=(\loc,\loc')$ such that $\vec{x}\models G(\tau)$. We put 
%$$\preexp{\lem}(\loc,\vec{x}):= \lem(\loc',\vec{x})$$.
%		\item 
%		If $\loc$ is an assignment location, then there is exactly one 
%transition $\tau=(\loc,\loc')$ outgoing from $\loc$.
%		We distinguish 
%		three cases, depending on $\updates(\tau)=(j,\up)$ (recall that $\up$ 
%is an update element):
%		\begin{compactitem}
%			\item If $\up\colon \Rset^{|\pvars|}\rightarrow \Rset$ is a 
%Borel-measurable function, then 
%			$$\preexp{\lem}(\loc,\vec{x}):= \lem(\loc',\vec{x}(j\leftarrow 
%\up(\vec{x}))).$$
%			\item If $\up$ is a distribution $d$, then $$ 
%\preexp{\lem}(\loc,\vec{x}):= 
%			\lem(\loc',\vec{x}(j\leftarrow \E[d])),$$ where $\E[d]$ is the 
%expected value of the 
%			distribution $d$.
%			\item 
%			If $\up$ is a set, then $$ \preexp{\lem}(\loc,\vec{x}):= 
%\sup_{a\in\up}
%			\lem(\loc',\vec{x}(j \leftarrow a)).$$
%		\end{compactitem}
%		
%%%$\mathrm{pre}_\eta(\loc,\mathbf{x}):=\eta(\loc',\expv_{\rvars}\left(f(\vec{x},\mathbf{r})\right))$
%		% if $\loc$ is a deterministic location, 
%%$(\loc,f,\loc')\in\transitions$ and 
%		%$\mathbf{x}\in G(\loc,f,\loc')$, where 
%		%$\expv_{\rvars}\left(f(\vec{x},\mathbf{r})\right)$ is the expected 
%%value of 
%		%$f(\vec{x},\cdot)$. %on $\rvars$ if $\vec{x}$ is treated as a constant 
%%%vector.
%	\end{compactitem}
%	%\end{definition}
%\end{definition}

%Intuitively, a pre-expectation of $\lem$ is a 
%function which for each configuration $(\loc,\vec{x})$ returns the maximal
%expected value of $\lem$ after one step is made from this configuration, where 
%the maximum is taken over all possible non-deterministic choices.
%Note that the only way in which pre-expectation can attain an infinite value 
%is 
%due to non-deterministic assignment. However, if the program in question has 
%\emph{bounded non-determinism}, in the sense that all non-deterministic 
%assignments can only choose the assignment value from a bounded set\footnote{A 
%set $u \subseteq \mathcal{R}$ is bounded if it is contained in some interval 
%of 
%a finite length}, then the pre-expectation of any real-valued measurable map 
%is 
%a real-valued function.

As in termination analysis of non-probabilistic programs, our LexRSMs are typically supported by \emph{invariants}, i.e. overapproximations of the set of reachable configuration. 

\begin{definition}[Invariant Map and Linear Invariant Map]
An \emph{invariant map} for a \PP{} $\program$ is a function $\inv$ assigning to each location of $\pCFG_{\program}$ a Borel-measurable set $\inv({\loc})\subseteq \Rset^{|\pvars|}$ of variable valuations, so called invariant of $\loc$, such that for each configuration $(\loc,\vec{x})$ reachable from the initial configuration it holds $\vec{x}\in \inv(\loc)$. Additionally, if each set $\inv(\loc)$ is of the form $\{\vec{x}\mid\vec{x}\models \Psi^\ell \}$ for some propositionally linear predicate $\Psi^\ell$, then we call $\inv$ a \emph{linear invariant map} (LIM).
\end{definition}

Slightly abusing the notation, we view each LIM equivalently as a function assigning linear predicates (whose satisfaction sets overapproximate the set of reachable valuations) to program locations.

We now have all the ingredients needed to define the notion of LexRSM maps for probabilistic programs. For notational convenience, we extend the function $\guards$ (which assigns guards to deterministic transitions) to the set of all generalized transition: for a generalized transition $\tau'$ which is not a standard transition outgoing from deterministic location, we put $\guards(\tau')=0\leq 0 \equiv \mathit{true}$.

\begin{definition}[Lexicographic Ranking Supermartingale Map]
Let $\eps>0$. An $n$-dimensional \emph{lexicographic $\eps$-ranking supermartingale map} ($\eps$-LexRSM map) for a program $\program$ supported by an invariant map $\inv$ is an $n$-dimensional measurable map $\vec{\lem}=(\lem_1,\dots,\lem_n)$ for $\program$ such that for each configuration $(\loc,\vec{x})$ where $\loc\neq \locterm$ and $\vec{x}\in \inv(\loc)$ the following conditions are satisfied:
 \begin{compactitem}
 	%\item For all $(\loc,\vec{x})\in \locs\times\Rset^{|\pvars|}$ it holds
 	%\item For all  it holds 
 	\item
 	for all $1\leq j \leq n$, $\lem_j(\loc,\vec{x})\geq 0$; and
 	\item 
 	for each generalized transition $\tilde{\tau}$ outgoing from $\loc$ such that $\vec{x}\models\guards(\tilde\tau)$ there 
 	exists $1\leq j 
 	\leq$ n such that
 	\begin{compactitem}
 	\item
 	$\tilde{\tau}$ is $\eps$-ranked by $\lem_j$ from $(\loc,\vec{x})$
 	\item
 	for all $1\leq j'<j$ we have that $\tilde{\tau}$ is unaffected by 
 	$\lem_{j'}$ from $(\loc,\vec{x})$.
 	\end{compactitem}
 \end{compactitem}
If additionally $\lem$ is a linear expression map, then we call it a linear $\eps$-LexRSM map ($\eps$-LinLexRSM).
\end{definition}

%\textbf{[PETR: DEFINE LOCATION TERMINATION]}

The main result is the soundness of $\eps$-LexRSM maps for proving a.s. 
termination.

\begin{theorem}
\label{thm:lexrsm-programs}
Let $\program$ be a probabilistic program. Assume that there exists an $\eps>0$ 
and an $n$-dimensional $\eps$-LexRSM map $\vec{\lem}=(\lem_1,\dots,\lem_n)$ for 
$\program$ supported 
by some 
invariant map $\inv$. 
Then $\program$ terminates almost surely.
\end{theorem}
\begin{proof}
Let $\sigma$ be any measurable scheduler and $\vecinit\in\vecinitset$ any 
initial variable valuation in $\program$.
We define an $n$-dimensional stochastic process 
$\{\vec{X}_{i}\}_{i=0}^{\infty} $ on the probability space 
$(\OmegaRun,\natfilt,\probm^{\sigma}_{\vecinit})$ such 
that for each 
$i\geq 0$ and $1\leq j 
\leq n$ and each run $\run$ we put $\vecseq{X}{i}{j}(\run) = 
\lem_j(\cfg{\sigma}{i}(\run))$. We claim that $\{\vec{X}_{i}\}_{i=0}^{\infty}$ 
is a strict $n$-dimensional $\eps$-LexRSM for the termination time $\ttime$ of $\program$. Clearly 
the process is real-valued, componentwise non-negative, and adapted to the 
canonical filtration of $\natfilt$. It remains to prove that condition (3) in 
Definition~\ref{def:lexrsm} is satisfied. To this end, for each $i\geq 0$ we 
define an almost-sure partition of the set $\{\run\in \OmegaRun\mid 
\ttime(\run) >i\}$ into sets $L^{i}_1,\dots,L^{i}_n$ by putting $L^i_j$ to be 
the set of all runs $\run$ such that $\ttime(\run)>i$ and for $\run$ the index 
$j$ is the smallest one such that the $(i+1)$-th transition on $\run$ is ranked 
by $\lem_j$ from $\cfg{\sigma}{i}(\run)$. Due to 
definition of an $\eps$-LexRSM map such a $j$ exists for all 
$\run\in\{\ttime>i\}$ and hence we indeed have a partition (so $L^i_{n+1}=\emptyset$ for all $i$). It remains to prove 
that irrespective of the initial choice of $\sigma$ and $\vecinit$ it holds, 
for each $1\leq j 
\leq n$ and $j'<j$, that $\E^\sigma_{\vecinit}[\vecseq{X}{i+1}{j}\mid 
\natfilt_i]\leq X_i[j]-\eps $ on $L_j^i$ and 
$\E^\sigma_{\vecinit}[\vecseq{X}{i+1}{j}\mid 
\natfilt_i]\leq X_i[j] $ on $L_{j'}^i$. This follows easily from 
the definition of $L^{i}_1,\dots,L^{i}_n$ and from the definition of a 
transition being $\eps$-ranked by $\lem_j$.

Since  $\{\vec{X}_{i}\}_{i=0}^{\infty}$ 
is an $\eps$-LexRSM for $\ttime$, from Theorem~\ref{thm:lexrsm-programs} it 
follows that $\probm^{\sigma}_{\vecinit}(\ttime <\infty) =1$, irrespective of 
$\sigma$ and $\vecinit$.
\end{proof} 

We conclude this section by showing that (Lin)LexRSMs can, unlike 1-dimensional 
RSMs, prove a.s. 
termination of programs whose expected termination time is infinite.
\lstset{language=affprob}
\lstset{tabsize=2,escapechar=\&}
\newsavebox{\infas}
\begin{lrbox}{\infas}
	\begin{lstlisting}[mathescape]
$x:=1$;$c:=1$ 
while $c\geq 1$ do				
	if prob(0.5) then			
		$x:=2\cdot x$			
	else						
		$c:=0$					
	fi							
od								
while $x\geq 0$ do $x:=x-1$ od	
	\end{lstlisting}
\end{lrbox}
\newsavebox{\infast}
\begin{lrbox}{\infast}
	\begin{lstlisting}[mathescape]

$(6c+2,x)$
$(6c+1,x)$
$(6c+3,x)$

$(3,x)$


$(0,x)$	
	\end{lstlisting}
\end{lrbox}
\newsavebox{\infastinv}
\begin{lrbox}{\infastinv}
	\begin{lstlisting}[mathescape]

$[c\geq 0 \wedge x\geq 1]$
$[c\geq 1 \wedge x\geq 1]$
$[c\geq 1 \wedge x\geq 1]$

$[c\geq 1 \wedge x \geq 1]$


$[x\geq 0]$	
	\end{lstlisting}
\end{lrbox}
\begin{figure}[t]
	\centering
	\usebox{\infas}
	\hspace{0.1cm}
	\usebox{\infast}
	\hspace{0.1cm}
	\usebox{\infastinv}
\caption{An a.s. terminating program with infinite expected termination time. A 
2-dimensional $1$-LinLexRSM map for the program is given on the right, along 
with the 
supporting invariants in square brackets. The invariants and a LinLexRSM on 
each 
line belong to the program location in which the program is \emph{before} 
executing the command on that line. The function is indeed a $1$-LinLexRSM, 
since in the probabilistic branching location $\loc$ we have 
$\preexp{\lem}(\loc,(x,c))=3c+3$ and $3c+3\leq 6c+1-1$ for all $c\geq 1$.} 
\label{fig:inftime}
\end{figure}

\begin{example}
\label{ex:infinite-time}
Consider the program in Figure~\ref{fig:inftime}. It is easy to see that it 
terminates a.s., but the expected termination time is infinite: to see this, 
note that that the expected value of variable $x$ upon reaching the second loop 
is $\frac{1}{2}\cdot 1 + \frac{1}{4}\cdot 2 + \frac{1}{8}\cdot 4 + \cdots = 
\frac{1}{2}+\frac{1}{2}+\frac{1}{2}+\cdots=\infty$ and that the time needed to 
get out of the second loop is equal to the value of $x$ upon entering the loop. 
However, a.s. termination of the program is proved by a 2-dimensional LinLexRSM 
$\lem$ pictured in the figure.
\end{example}



